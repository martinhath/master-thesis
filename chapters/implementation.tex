\chapter{Implementation\label{ch:implementation}}

In this chapter we look at the Rust implementation of CMR\@. The source code is openly available
on GitHub under \todo{license}{\color{red}some license}\cite{cmr-github}.

We start this chapter by implementing the primitive types and operations from
\cref{sec:cmr-primitives}.




\section{Data}

In order to better understand how CMR is laid out, we start out by looking at the data.
\todo{flowcharts fred brooks}

Allocated addresses are stored in a global \code{HashSet}, \code{ALLOCS}, which uses a \code{Mutex}
for thread synchronization. Only addresses in \code{ALLOCS} are subject for reclamation.


Thread also stores data in \gls{tls}. Each thread maintains a \code{Vec} of pointers to their
\code{Guard}s, such that collecting all guards is just a matter of iterating through the
\code{Vec}, and following the pointer. Since the data is thread local, no synchronization is needed
when operating on the \code{Vec}, which makes updates cheap.

One caveat of CMR is that new threads needs to register themselves before using the system. This is
done through \code{cmr::register_thread()}. This initializes thread local data, and pushes a
thread handle used in \cref{sec:impl-snapshot}.



\section{Primitives}

Blabla

\subsubsection{Guard}

The \code{Guard} is implemented as a single word, in addition to an empty type (the
\code{PhantomData}) as Rust \emph{requires} generic types to be used. \code{Guard}s aren't normally
constructed directly (\fixme{non movable types somewhere}), but rather declared with the
\code{guard!} macro, which constructs it and calls \code{Guard::register}.
An excerpt of the definitions of \code{Guard} is shown in \cref{lst:guard-impl}.

\begin{figure}
\begin{lstlisting}[caption=Excerpt of \mc{Guard}s definitions,label=lst:guard-impl]
struct Guard<T> {
  ptr: usize,
  _marker: PhantomData<T>,
}
impl<T: Trace> Guard<T> {
    pub unsafe fn new() -> Self { Guard { inner: 0, _marker: PhantomData, } }
    pub fn copy_guard(&mut self, other: &Self) { self.inner = other.inner; }
    pub fn register(&mut self) {
        ROOTS.with(|r| { let mut v = r.borrow_mut();
                         v.push(GuardPointer::from_guard(self)) });
    }
    ...
}
macro_rules! guard {
    ($var:ident) => { let $var = unsafe { &mut $crate::guard::Guard::new() };
                      $var.register();
    }
}\end{lstlisting}

\end{figure}

\code{Guard::register} gets a mutable reference to the thread local \code{Vec} of \code{Guard}s,
and inserts a pointer to itself into it. \code{Guard::drop} (omitted) does the opposite.
Note that we insert a \code{GuardPointer}; we will revisit this in \cref{sec:impl-tracking-types}.
\code{Guard::new} is marked \code{unsafe} since the caller must guarantee to \code{register} the
guard before using it. This is normally handled by the \code{guard!} macro, but there are use cases
for calling \code{new} directly. Usage of the guard is normally as follows:
\begin{lstlisting}[style=Rust]
{
  guard!(g);
  let my_num = cmr::alloc(123, g);
} // `g` is dropped here
\end{lstlisting}
Note that by using \code{guard!}, the caller only obtains a \code{\&mut Guard<T>}, and not the
\code{Guard<T>} itself; this makes it impossible to move the \code{Guard} in memory.


\subsubsection{Atomic}

\code{Atomic} is mainly a wrapper around Rusts \code{AtomicPtr}, although the internals differ
slightly. CMR defines its own type so that we can control the return types of certain functions.
\cref{lst:def-atomic} shows the definition of the \code{struct}, as well as \code{cas}, the
compare-and-swap operation, in which we utilize some Traits from the Rust standard library to
convert between types.  Implementation of remaining methods are straight forward.

\begin{lstlisting}[
  style=Rust,
  caption=Excerpt of \mc{Atomic}s definitions (Trait bounds omitted for brevity),
label=lst:def-atomic]
pub struct Atomic<T> { data: AtomicUsize, _marker: PhantomData<T>, }
impl<T> Atomic<T> {
    pub fn cas<'a, A, B>(&self, a: A, b: B, ordering: Ordering)
     -> Result<A, NullablePtr<'a, T>> {
        let (old, new) = (raw(a), raw(b));
        let ret = self.data.compare_and_swap(old, new, ordering);
        if ret == old { Ok(A::try_from(NullablePtr::new(ret)).unwrap()) }
        else { Err(NullablePtr::new(ret)) } }
    (*\lit{\dots Remaining methods}*) }
\end{lstlisting}



\subsubsection{NullablePtr}

\code{NullablePtr} is used as the canonical pointer type in CMR, and all pointer like types are
converted to \code{NullablePtr} using the \code{From} and \code{Into} traits from the Rust standard
library, which handles coversion between types. For instance, we implement \code{From<*const T> for
NullablePtr<T>}. This way we can write functions that are generic over all types of pointers, so
that the user of CMR does not have to handle these conversions themselves.


The definition of \code{NullablePtr} is shown in \cref{lst:def-nullableptr}, with the \code{new}
and \code{ptr} methods. Note that we cannot get a reference to the \code{T} that \code{NullablePtr}
points to; this is because we don't know if the pointer is \code{null} or not. \code{ptr} promotes
the \code{NullablePtr} to a \code{Ptr}, should it not be \code{null}, by using the \code{Option}
type which Rust provides.

\begin{figure}[ht]
  \begin{lstlisting}[style=Rust,label=lst:def-nullableptr,
  caption=Definition of \code{NullablePtr}]
pub struct NullablePtr<'a, T: 'a>(usize, PhantomData<&'a T>);
impl<'a, T> NullablePtr<'a, T> {
    pub fn new(u: usize) -> Self { NullablePtr(u, PhantomData) }
    pub fn ptr(self) -> Option<Ptr<'a, T>> {
        if addr(self) == 0 { None }
        else { unsafe { Some(Ptr::new(raw(self))) } }
    (*\lit{\dots Remaining methods}*) }
\end{lstlisting}
\end{figure}


\subsubsection{Ptr}

\code{Ptr} provides access to the type it points to, as it is guaranteed to be non-\code{null}.
This is done through the \code{Deref} trait, which handles the \code{*} in Rust. Due to auto-deref,
we can now use \code{\&Ptr<T>} in place of a \code{\&T}. The definition of \code{Ptr}, a new of its
methods, and its \code{Deref} implementation is shown in \cref{lst:def-ptr}. Node that both
\code{new} and \code{get_mut} are \code{unsafe} methods; \code{new} because we cann't guarantee
that the address passed is valid, and \code{get_mut} because the data may be aliased.

\begin{figure}[ht]
  \begin{lstlisting}[style=Rust,label=lst:def-ptr,
  caption=Definition of \code{Ptr},
  escapeinside={(@}{@)},
  ]
pub struct Ptr<'a, T: 'a> { data: usize, _marker: PhantomData<&'a T> }
impl<'a, T> Ptr<'a, T> {
    pub(crate) unsafe fn new(u: usize) -> Self {
        Self { data: u, _marker: PhantomData, } }
    pub unsafe fn get_mut(&mut self) -> &mut T { &mut *self.as_raw() }
    fn as_raw(&self) -> *mut T { with_tag(*self, 0).data as *mut T }
    (@\lit{\dots Remaining methods}@) }
impl<'a, T> Deref for Ptr<'a, T> {
    type Target = T;
    fn deref(&self) -> &T { unsafe { &*(self.as_raw()) } } }
\end{lstlisting}
\end{figure}


\subsubsection{Tagging}

By having one canonnical pointer type, we can define functions that are generic over all types
that supports conversion from and/or to \code{NullablePtr}. This is used in the functions for
pointer tagging, as well as the \code{cas} in \cref{lst:def-atomic} (the types \code{A} and
\code{B}). \cref{lst:tagging} shows some of the free functions for pointer tags that are generic
over different pointer types.

\begin{figure}[ht]
\begin{lstlisting}[
  numbers=left, numberstyle=\color{gray}\ttfamily{}TA,
  caption=Implementation of pointer tagging functions,
  label=lst:tagging]
pub fn tag<'a, P, T: 'a>(p: P) -> usize where P: Into<NullablePtr<'a, T>> {
    let n: NullablePtr<T> = p.into();
    n.0 & ones(TAG_BITS) }

pub fn with_tag<'a, P, T: 'a>(p: P, tag: usize) -> P
where P: Into<NullablePtr<'a, T>> + TryFrom<NullablePtr<'a, T>> {
    let p = p.into();
    let n = (p.0 & !(ones(TAG_BITS))) | tag;
    P::try_from(NullablePtr::new(n)).unwrap_or_else(|_e| panic!("failed conversion")) }


\end{lstlisting}

\end{figure}

\code{ones(k)} returns the bitmask with the \code{k} lower bits set, and \code{TAG_BITS} is a
predefined number of bits allowed to use for tagging for any pointer. We convert from \code{P} to
\code{NullablePtr} with \code{.into()} \coderef{TA2}.  In \code{with_tag} \coderef{TA5} we need
to use \code{TryFrom}, which is a conversion trait that may fail. In CMR \code{Ptr<T>} implements
\code{TryFrom<NullablePtr>}, where the conversion fails if the \code{NullablePtr} is \code{null}. We
assert that this failure should never happen \coderef{TA9} with the rationale that if we converted
some type \code{P} into a \code{NullablePtr} and changed its tag, we should be able to convert back
to \code{P}, even though the conversion is not always possible in general.



\section{Snapshot\label{sec:impl-snapshot}}

For obtaining a snapshot of the process memory CMR utilizses a operating system features offered
by POSIX compliant systems: \emph{forking}.
Calling \code{fork()} makes a copy of the current process, called the \emph{child process}.
The return value of \code{fork()} determined whether we are in the child or parent process.
In the child process, only the thread that called \code{fork()} continues its execution. For this
reason, we need to perform some work before forking. Most importantly, the threads needs to tell
the reclaiming thread where to find their \code{Guard}s. To do this we use a second POSIX feature:
\emph{signals} (see \cref{sec:background-signals}).

\begin{figure}[ht]
  \begin{lstlisting}[style=Rust,label=lst:thread-signaling,
  caption=Thread signaling,
  ]
fn signal_threads_except_self() -> usize {
    let mut count = 0;
    let me = thread_id();
    let mut th = THREAD_HANDLERS.lock().unwrap();
    th.retain(|&th|
      if th == me { true }
      else { unsafe {
            let val = libc::sigval { sival_ptr: std::ptr::null_mut() };
            let r = libc::pthread_sigqueue(th as u64, libc::SIGUSR1, val);
            if r == 0 { count += 1; true }
            else { false } } });
    count }
\end{lstlisting}
\end{figure}

Using \gls{pthreads} signals we register a signal handler for the \code{SIGUSR1} signal with the
\code{sigaction} call, and the reclaiming thread signals all threads with \code{pthread_sigqueue}.
This is done through the Rust library \code{libc}, which provides Rust bindings to the C standard
library. The procedure for signaling all registered threads is shown in
\cref{lst:thread-signaling}. Here we actually do two things at once: in addition to signaling the
threads, we remove the thread handlers that we fail to signal. The procedure returns the number of
threads we successfully signaled, so the caller knows how many threads to expect being in the
signal handler. Pseudo code for the signal handler is shown in \cref{lst:signal-handler}.

\begin{figure}[ht]
  \begin{lstlisting}[
    style=Rust,
    numbers=left,
    numberstyle=\color{gray}\ttfamily{}SH,
    caption=Pseudocode for the signal handler used by CMR,
    label=lst:signal-handler,
  ]
id = sh_enter_counter.fetch_add(1)  (*\label{c:sh-reg}*)
write_out_data_to(thread_datas[id]) (*\label{c:sh-write}*)
sh_done_counter.fetch_add(1)        (*\label{c:sh-done}*)
while sh_frozen.load():             (*\label{c:sh-while}*)
    wait()                          (*\label{c:sh-wait}*)
sh_enter_counter.fetch_sub(1)       (*\label{c:sh-leave}*)
\end{lstlisting}
\end{figure}


We use \code{sh_enter_counter} to keep track of how many threads are present in the signal handler;
the reclaimer knows how many threads it succesfully signaled, so it knows how many threads to
expect. \coderef{SH\ref*{c:sh-reg}} registers a threads presence, in addition to giving each thread
a unique index in the range $[0, n)$, where $n$ is the number of threads signaled. This is used
in~\coderef{SH\ref*{c:sh-write}}, where each thread writes out their guards and allocations into
the global vector \code{thread_datas}. We then register that we have written our
data~\coderef{SH\ref*{c:sh-done}}, and wait for the reclaimer to unfreeze
us~\coderef{SH\ref*{c:sh-wait}}. At last we register that we have seen that we are done
\coderef{SH\ref*{c:sh-leave}}, so that no thread risk being stuck in the next iteration of the
reclaming procedure, waiting again on the \code{sh_frozen} flag.



\section{Reachability}

Reachability analysis is a straight forward implementation of the \proc{Find-Reachable} procedure
from \cref{sec:cmr-overview}, and is shown in \cref{lst:impl-find-reachable}. We maintain one
\code{HashMap} \coderef{FR\ref*{c:fr-mk-seen}} for all blocks we have seen, and a \code{VecDeque}
\coderef{FR\ref*{c:fr-mk-queue}} for the queue of blocks we want to visit. For efficiency reasons
we write out the reachable set when we find a new block, instead of collecting up the blocks and
writing it in one iteration \coderef{FR\ref*{c:fr-write}}.  For this implementation we capped the
number of pointers a single type can write out to be 32 \coderef{FR\ref*{c:fr-buffer}}; while this
is not optimal in the general case, most types only require one or two pointers.



\begin{figure}[ht]
  \begin{lstlisting}[
    style=Rust,
    numbers=left,
    numberstyle=\color{gray}\ttfamily{}FR,
    caption=Rust implementation of \proc{Find-Reachable},
    label=lst:impl-find-reachable,
  ]
fn mark_and_sweep(mut cursor: Cursor<&mut [u8]>, roots: Vec<TraitObject>) -> usize {
    let mut seen = HashMap::new(); (*\label{c:fr-mk-seen}*)
    let mut queue = VecDeque::new(); (*\label{c:fr-mk-queue}*)
    (*\lit{insert roots into \code{seen} and \code{queue}}*)
    let mut num_ptr = 0;
    let mut ptr_buffer: [TraitObject; 32] = unsafe { std::mem::zeroed() }; (*\label{c:fr-buffer}*)
    while let Some(to) = queue.pop_front() {
        let addr = to.data as usize;
        let t: &ptr::Trace = unsafe { ::std::mem::transmute(to) };
        (*\lit{write out \code{addr} to the \code{cursor}}\label{c:fr-write}*)
        num_ptr += 1;
        let n = t.write(&mut ptr_buffer);
        for i in 0..n {
            let (to, addr, vtable) = (*\lit{destructure \code{ptr_buffer[i]}}*)
            if seen.insert(addr, vtable).is_none() {
                queue.push_back(to); } } }
    num_ptr }
\end{lstlisting}
\end{figure}


\subsection{Trace}

Finding pointers in arbitrary data types might involve significant work since the size of the data
types can be arbitrarily large. In addition, memory might not be initialized, and false positives
might occur if we are not careful. Instead of scanning through the memory block linearily, CMR
defines the Trait called \code{Trace}, which all data types that is in shared memory must
implement.  A type implementing \code{Trace} knows a bound on how many shared memory pointers it
contains, and can write these out to a buffer. For instance, a \code{Node} in a single linked list
contains only one pointer, namely its next pointer, which is trivial to write out.

The implementation of this uses \emph{Trait Objects} (\cref{sec:trait-objects}), which involves
dynamic dispatch. This solution is potentially expensive, as it may involve cahce misses in the
I-cache, although the number of misses is limited by the difference in data types in shared memory,
which normally is smaller than in Rust memory.

\begin{figure}[ht]
  \begin{lstlisting}[
  style=Rust,
  caption=Definition of the \code{Trace} trait and a sample implementation for a
  linked list node. The impementation uses \emph{specialization} (\cref{sec:specialization})
  as the implementation of \code{Node}s containing data that itself is \code{Trace} is different.,
  numbers=left,
  numberstyle=\color{gray}\ttfamily{}T,
  ]
pub trait Trace { fn count(&self) -> usize { 0 }
                  fn write(&self, &mut [TraitObject]) -> usize { 0 } }
pub struct Node<T> { data: ManuallyDrop<T>,
                     next: Atomic<Node<T>> }
impl<T> cmr::Trace for Node<T> {
    default fn count(&self) -> usize { 1 }
    default fn write(&self, slice: &mut [TraitObject]) -> usize {
        let p = unsafe { self.next.load(SeqCst) };
        if !p.is_null() { slice[0] = ptr::trait_object(p);
                          1 } 
        else { 0 } } }\end{lstlisting}

\end{figure}

\code{Trace} contains default implementations of the two functions, such that primitive types can
easily implement it. \code{write} takes a buffer, writes all pointers to it as \code{TraitObject}s,
and returns the number of objects written. \code{count} gives an upper bound on the number of
pointers written. This is useful for collection types, like \code{Vec} or \code{HashMap}, which
also may contain pointers to shared memory.

\code{Node} is a standard node from a linked list, containing \code{data}, and a \code{next}
pointer. The implementation of \code{write} loads the \code{next} pointer~\coderef{T12}, which is
an \code{unsafe} operation, as there is no \code{Guard} protecting the pointer. This is safe in the
context of the reclaimer since the memory will be freed at earliest when we finish the reachability
analysis, and at that point we no longer read the memory. The implementation only writes out the
pointer if it is non-null. While this is not required for CMR to function, it simplifies the logic
in the reachability analysis.


\section{Tracking Types\label{sec:impl-tracking-types}}

\todo{03/06 14:14 GuardPointer, DynDispatch?}


\section{Other Things}
\todo{name}

In addition to the types member function, we have defined two important free functions:
\code{guard} and \code{alloc}.
\code{guard} takes a \code{\&mut Guard} and \code{\&Atomic}, \code{load}s the atomic, and protects
the pointer read in the \code{Guard}, without a reclamation pass happening in between --- this is
ensured by the \code{atomic} call.

\begin{lstlisting}[style=Rust]
pub fn guard<'a, T>(guard: &'a mut Guard<T>, a: &Atomic<T>) -> NullablePtr<'a, T> {
    without_reclamation(|| { let p = unsafe { a.load(SeqCst) };
                             guard.inner = ptr::raw(p);
                             p }) }
\end{lstlisting}
\code{alloc} is similar to \code{Box::new}, as it just wraps \code{cmr::alloc::alloc} in addition
to protecting the pointer returned in the \code{Guard} passed in.
\begin{lstlisting}[style=Rust,numbers=left, numberstyle=\color{gray}\ttfamily{}A]
pub fn alloc<T: Trace>(guard: &mut Guard<T>, t: T) -> Ptr<T> {
    let ptr = alloc::alloc(t);
    guard.inner = ptr::addr(ptr);
    alloc::register(ptr);
    ptr }
\end{lstlisting}


\section{Correctness}
\todo{name}

\begin{claim}
  \code{Guard} satisfies \cref{def:guard}.
\end{claim}






\section{Complications}

\subsection{Locks in libc\label{sec:alloc-lock}}

In order to protect programmers from deadlocks, POSIX defines a subset of functions as
\emph{async-signal safe}, meaning they are safe to call from a signal handler. Functions that are
async-signal safe includes \code{time()}, \code{open()}, and \code{mkdir()}, but it does \emph{not}
include \code{malloc()}. As such, allocation in signal handlers is not safe, and is a source of
deadlock bugs. This itself was not a large problem for CMR, as its signal handler did not require
any allocation. However, as threads are frozen by the reclaimer in a signal handler, it is also not
safe for the \emph{reclaimer} to call \code{malloc}, despite not being in a signal handler itself.
This is because some thread may be in the process of allocating memory, and have aquired a lock
internal to libc, right before being signaled. The thread is still holding the libc lock and is
frozen in its signal handler by the consolidator, which prevents \emph{all threads}, even those
oblivious to CMR, from allocating.

This problem is not solved properly by CMR, but its effects are mitigated by wrapping the general
allocator in Rust to go through yet another lock, the \code{alloc_lock}, which can be aquired by
the reclamating thread (this is why we have~\coderef{R3} in \cref{lst:reclaimer}). This
prevents most allocations of deadlocking, but not all. Rust uses \gls{pthreads} internally for thread
handling on Linux, which allocates internally, both in \code{spawn} and \code{join}. The former may
be circumvented by aquiring the allocation lock before calling it, but this is no solution for the
latter, since the thread may depend on allocating before exiting.
\fixme{deactivate/reactivate around joins?}

\fixme{add example or something here}

\subsection{SignalVec}

\subsection{Thread Registration}









% \clearpage
% \clearpage
% \clearpage
%
% Recall the \proc{Reclaim} procedure from \cref{ch:cmr}. In \cref{sec:impl-snapshot} we look at how
% the snapshot mechanism is implemented. \cref{sec:impl-thread-data} discusses
%
% We look at four things: how thread
% freezing is implemented (\cref{sec:signals}), how reachability analysis is done
% (\cref{sec:reachability}), thread and process intercommunication
% (\cref{sec:thread-communication}), and how the primitives and operations as defiend in
% \cref{ch:cmr} are implemented (\cref{sec:data-types}).
%
%
% \section{The Reclaimer\label{sec:thread-communication}}
%
% There is only at most one reclaimer at any time. The first thing a potential reclaiming thread
% tries to do, is take the lock~\coderef{R2}. Then we take another lock that is used for
% allocation~\coderef{R3} (see \cref{sec:alloc-lock}). This may block, but since we have not
% frozen any threads yet, any thread holding the lock will eventually release it, assuming it itself
% is not deadlocked. Next we initialize shared variables, such as the counters used by threads in
% their signal handler. After they are initialized, we freeze all threads~\coderef{R5}, and count how
% many threads were successfully frozen, which is used in~\coderef{R6} where we wait on all threads
% to finish their job. When done, the data that the threads write out, their \code{Guard}s and
% allocations, are read~\coderef{R7}.
%
% \begin{figure}[ht]
%   \begin{lstlisting}[
%     style=Rust,
%     numbers=left,
%     numberstyle=\color{gray}\ttfamily{}R,
%     caption=Pseudocode of the work of the reclaimer,
%     label=lst:reclaimer,
%   ]
% run_reclaim_pass() {
%     if reclaim_lock.lock().fail() { return }
%     lock_malloc()
%     init_shared_vars()
%     n = freeze_threads()
%     while sh_done_counter.load() != n { wait() }
%     (guards, allocs) = read_thread_datas()
%     memory_map = mmap()
%     write_marker(memory_map)
%     if fork() == Child {
%         rs = find_reachable()
%         write_to_mmap(rs, memory_map)
%         exit(0) }
%     sh_frozen.store(false)
%     send_to_background_thread(allocs, memory_map) }\end{lstlisting}
% \end{figure}
%
% When \code{fork()}ing the child process continues the thread of the parent process that called
% \code{fork()}, such that it has access to everything that the original thread had\fixme{rewrite}.
% As such, we don't need to communicate from parent to child. However, the job of the child process
% is to run reachability analysis, and we do need its result. CMR uses \emph{memory map}
% (\cref{sec:memory-map}) for \gls{ipc}\@, which is set up before the fork. Since we need to
% know when the child is done writing its results, we write a \emph{marker} word as the first word in
% the memory map. Then we \code{fork()}. The child process does the reachability analysis, and
% writes the result after marker in the memory map. When it is done writing, it overwrites the marker
% with the number of elements written.
%
% \begin{figure}[b]
%   \centering
%   \begin{tikzpicture}[node distance=0.7cm]
  \tikzstyle{mem} = [draw, fill=white, outer sep=0mm, minimum width=1cm,
                     minimum height=0.5cm, node distance=0.5cm];
  \tikzstyle{parent} = [fill=white,draw,minimum width=3.5cm]

  \node (parent-label) {Parent Process};
  \node [parent] (t1) [below of=parent-label] {$T\sb{1}$};
  \node [parent] (t2) [below of=t1]           {$T\sb{2}$};
  \node [parent] (t3) [below of=t2]           {$T\sb{3}$};
  \node [parent] (bt) [below of=t3]           {Background Thread};

  \node [fill=white,draw,minimum width=2cm, node distance=7cm] (t'2) [right of=t2] {$T'\sb{2}$};
  \node (child-label) [above of=t'2] {Child Process};

  \node [node distance=4cm] (mmap-label) [right of=parent-label] {\code{mmap}};
  \node [mem] (m1) [below of=mmap-label] {\code{0xbeef}};
  \node [mem] (m2) [below of=m1]         {\code{0xcafe}};
  \node [mem] (m3) [below of=m2]         {\code{0xbabe}};
  \node [mem] (m4) [below of=m3]         {\code{0xdead}};
  \node [mem] (m5) [below of=m4]         {\code{0xabba}};

  \begin{scope}[on background layer]
    \node [draw,fit={(parent-label) (bt)},                 fill=gray!15] {};
    \node [draw,fit={(child-label) (t'2)},                 fill=gray!15] {};
    \node [draw,fit={(mmap-label) (m5)},inner sep=0mm,fill=gray!15] {};
  \end{scope}

  \draw[-latex] (bt.east) -| ($(bt.east)!0.5!(m1.west)$) |- (m1.west);
  \draw[-latex] (t'2.west) -| ($(t'2.west)!0.5!(m4.east)$) |- (m4.east);
\end{tikzpicture}

%   \caption{Illustration of \gls{ipc} through a memory map. $T\sb{2}$ in the parent process is the
%   reclaiming thread, so $T\sb{2}$ is the one thread in the child process. Both processes have
%   access to the same memory in the memory map.}
% \end{figure}
%
% In order to minimize the delay of the reclamation pass from the point of view of the user, we spawn
% a background thread which handles the part of the reclamation performed after we \code{fork()}.
% This ensures that no user thread has to wait for the child process to finish its reachability
% analysis before going back to running application code.  After forking, the parent process
% unfreezes the other threads, and sends all necessary data to the background thread. Note that the
% \code{reclaim_lock} is not released, even though the thread is exiting the procedure. The lock is
% only released when the background thread is fully done with the reclamation pass.
%
% \section{Thread Freezing\label{sec:signals}}
%
% \emph{Signals} is a process communication mechanism used by POSIX compliant operating systems.
% \gls{pthreads} also supports signaling for communication between threads in a
% process. We utilize this in order to implement thread freezing, by registering a signal handler for
% the signal \code{SIGUSR1}. Pseudocode for the signal handler CMR uses is shown in
% \cref{lst:guard-impl}. The signal handler is registered with the \code{sigaction}
% function, and threads are signaled with \code{pthread_sigqueue}.
%
% \begin{figure}[b]
%   \begin{lstlisting}[
%     style=Rust,
%     numbers=left,
%     numberstyle=\color{gray}\ttfamily{}SH,
%     caption=Pseudocode for the signal handler used by CMR,
%     label=lst:signal-handler,
%   ]
% if is_in_cleanup:
%     clean_up_and_return()
% id = sh_enter_counter.fetch_add(1)
% write_out_data_to(thread_datas[id])
% sh_done_counter.fetch_add(1)
% while sh_frozen.load():
%     wait()
% sh_enter_counter.fetch_sub(1)\end{lstlisting}
% \end{figure}
%
% We use \code{sh_enter_counter} to keep track of how many threads are present in the signal handler;
% the reclaimer knows how many threads it succesfully signaled, so it knows how many threads to
% expect. \coderef{SH3} registers a threads presence, in addition to giving each thread a unique
% index in the range $[0, n)$, where $n$ is the number of threads signaled. This is used
% in~\coderef{SH4}, where each thread writes out their guards and allocations into the global vector
% \code{thread_datas}. We then register that we have written our data~\coderef{SH5}, and wait for the
% reclaimer to unfreeze us~\coderef{SH6}.
% \fixme{Since we don't know that we have exited the handler after SH6, do we really need SH7??}
%
%
%
% The signal handler is registered by a thread with a call to the \code{thread_activate} procedure,
% which must be called before the thread is using CMR\@. \fixme{bench automation of this in Guard or
% something?}
% The system has a fixed upper bound on the number of threads that may be active at any time, through
% the \code{thread_datas} vector, which is pre-allocated with a fixed size \fixme{check this out}.
%
% Since a thread handler may have registered it leaving by decrementing \\\code{sh_enter_counter},
% but then preempted before leaving the handler, CMR uses \\\code{pthread_sigqueue} instead of the more
% standard \code{pthread_kill}, as \gls{pthreads} may simply ignore a signal that is sent to a
% thread when that thread is already in the signal handler for a given signal.
% This is problematic, since the reclaimer keeps carefully track of where threads are in their
% handler, and waits for all threads to reach certain checkpoints. If the reclaimer is lead to
% believe that a thread just entered the handler when in fact it was just leaving, it will lead to a
% deadlock.
%
%
% \section{Reachability\label{sec:reachability}}
%
% Reachability analysis is done through a standard mark-and-sweep algorithm. The reclaimer collects
% all roots from all active threads, and looks through the memory of each data type that is pointer
% to by the roots. If any data type contains a \code{Guard} or \code{Ptr}, the address is marked as seen,
% and added to the frontier. When the frontier is empty, we have registered all reachable memory.
%
% Finding pointers in arbitrary data types might involve significant work since the size of the data
% types can be arbitrarily large. in addition to that memory might not be initialized, and false
% positives. Instead of scanning through the memory block linearily, \fixme{should write somewhere why
% we can assume that the ptr is the start of a block} CMR has a Trait called \code{Trace}, which all
% data types that is in shared memory must implement.  A type implementing \code{Trace} knows a bound
% on how many shared memory pointers it contains, and can write these out to a buffer. For instance,
% a \code{Node} in a single linked list contains only one pointer, namely its next pointer, which is
% trivial to write out.
%
% The implementation of this uses \emph{Trait Objects} (\cref{sec:trait-objects}), which involves
% dynamic dispatch. This solution is potentially expensive, as it may involve cahce misses in the
% I-cache, although the number of misses is limited by the difference in data types in shared memory,
% which normally is smaller than in Rust memory.
%
% \begin{figure}[ht]
%   \begin{lstlisting}[
  style=Rust,
  caption=Definition of the \code{Trace} trait and a sample implementation for a
  linked list node. The impementation uses \emph{specialization} (\cref{sec:specialization})
  as the implementation of \code{Node}s containing data that itself is \code{Trace} is different.,
  numbers=left,
  numberstyle=\color{gray}\ttfamily{}T,
  ]
pub trait Trace { fn count(&self) -> usize { 0 }
                  fn write(&self, &mut [TraitObject]) -> usize { 0 } }
pub struct Node<T> { data: ManuallyDrop<T>,
                     next: Atomic<Node<T>> }
impl<T> cmr::Trace for Node<T> {
    default fn count(&self) -> usize { 1 }
    default fn write(&self, slice: &mut [TraitObject]) -> usize {
        let p = unsafe { self.next.load(SeqCst) };
        if !p.is_null() { slice[0] = ptr::trait_object(p);
                          1 } 
        else { 0 } } }\end{lstlisting}

% \end{figure}
%
% \code{Trace} contains default implementations of the two functions, such that primitive types can
% easily implement it. \code{write} takes a buffer, writes all pointers to it as \code{TraitObject}s,
% and returns the number of objects written. \code{count} gives an upper bound on the number of
% pointers written. This is useful for collection types, like \code{Vec} or \code{HashMap}, which
% also may contain pointers to shared memory.
%
% \code{Node} is a standard node from a linked list, containing \code{data}, and a \code{next}
% pointer. The implementation of \code{write} loads the \code{next} pointer~\coderef{T12}, which is
% an \code{unsafe} operation, as there is no \code{Guard} protecting the pointer. This is safe in the
% context of the reclaimer since the memory will be freed at earliest when we finish the reachability
% analysis, and at that point we no longer read the memory. The implementation only writes out the
% pointer if it is non-null. While this is not required for CMR to function, it simplifies the logic
% in the reachability analysis.
%
%
% \section{Data Types\label{sec:data-types}}
%
% In this section we look at the concrete Rust implementation of selected data types and operations,
% and argue for their correctness.
%
%
% \subsubsection{Guard}
%
% The \code{Guard} is implemented as a single word, in addition to an empty type (the
% \code{PhantomData}) as Rust \emph{requires} generic types to be used. \code{Guard}s aren't normally
% constructed directly (\fixme{non movable types somewhere}), but rather declared with the
% \code{guard!} macro, which constructs it and calls \code{Guard::register}.
% An excerpt of the definitions of \code{Guard} is shown in \cref{lst:guard-impl}.
%
% \begin{lstlisting}[caption=Excerpt of \mc{Guard}s definitions,label=lst:guard-impl]
struct Guard<T> {
  ptr: usize,
  _marker: PhantomData<T>,
}
impl<T: Trace> Guard<T> {
    pub unsafe fn new() -> Self { Guard { inner: 0, _marker: PhantomData, } }
    pub fn copy_guard(&mut self, other: &Self) { self.inner = other.inner; }
    pub fn register(&mut self) {
        ROOTS.with(|r| { let mut v = r.borrow_mut();
                         v.push(GuardPointer::from_guard(self)) });
    }
    ...
}
macro_rules! guard {
    ($var:ident) => { let $var = unsafe { &mut $crate::guard::Guard::new() };
                      $var.register();
    }
}\end{lstlisting}

%
% \fixme{rewrite, focus on \code{register}}
%
% Since we need to keep track of type information dynamically, we construct a \code{GuardPointer}
% which does exactly this, and pushes it onto a thread local \code{Vec}, \code{ROOTS}.  It is also
% possible to construct a \code{NullablePtr} from the \code{Guard}.  The \code{guard!} macro corresponds
% to make-guard (\cref{eq:make-guard}). copy-guard (\cref{eq:copy-guard}) is shown in the Listing.
%
% Using a macro for declaring \code{Guard}s is a trick that serves two purposes:
% \begin{enumerate*}[1) ]
%   \item the user needs to write \code{unsafe} in order to make a \code{Guard} without calling
%     \code{register} on it,
%   and\item the user never gets a binding to a variable of type \code{Guard}, but only of type
%     \code{\&mut Guard}, which prevents moving the \code{Guard}, hence fulfulling the requirement of a
%     \code{Guard} being non-movable from \cref{def:guard}.
% \end{enumerate*}
%
% In addition to the \code{guard!} macro for declaring new \code{Guard}s, CMR also defines the
% \code{guard} \emph{procedure}, which loads an \code{Atomic} into a \code{Guard}.
%
%
% \subsubsection{Atomic}
%
% \code{Atomic} is mainly a wrapper around Rusts \code{AtomicPtr}, although the internals differ
% slightly. CMR defines its own type so that we can defide on the return types of certain functions.
% We include the definition of the \code{struct}, as well as \code{cas}, the compare-and-swap
% operation, in which we utilize some Traits from the Rust standard library to convert between types.
% \fixme{write a little here}
%
% \begin{lstlisting}[
  style=Rust,
  caption=Excerpt of \mc{Atomic}s definitions (Trait bounds omitted for brevity),
label=lst:def-atomic]
pub struct Atomic<T> { data: AtomicUsize, _marker: PhantomData<T>, }
impl<T> Atomic<T> {
    pub fn cas<'a, A, B>(&self, a: A, b: B, ordering: Ordering)
     -> Result<A, NullablePtr<'a, T>> {
        let (old, new) = (raw(a), raw(b));
        let ret = self.data.compare_and_swap(old, new, ordering);
        if ret == old { Ok(A::try_from(NullablePtr::new(ret)).unwrap()) }
        else { Err(NullablePtr::new(ret)) } }
    (*\lit{\dots Remaining methods}*) }
\end{lstlisting}

%
%
% \subsubsection{NullablePtr}
%
% \code{NullablePtr} is used as the canonical pointer type in CMR, and all pointer like types are
% converted to \code{NullablePtr} using the \code{From} \code{Into} traits, which
% handles coversion between types. For instance, we implement \code{From<*const T> for
% NullablePtr<T>}. This way we can write functions that are generic over all types of pointers, so
% that the user of CMR does not have to handle these conversions themselves. This is used in the
% functions for pointer tagging, as well as the \code{cas} in \cref{lst:atomic} (the types \code{A}
% and \code{B}). \cref{lst:tagging} shows some of the free functions for pointer tags that are
% generic over different pointer types.
%
% \begin{lstlisting}[
  numbers=left, numberstyle=\color{gray}\ttfamily{}TA,
  caption=Implementation of pointer tagging functions,
  label=lst:tagging]
pub fn tag<'a, P, T: 'a>(p: P) -> usize where P: Into<NullablePtr<'a, T>> {
    let n: NullablePtr<T> = p.into();
    n.0 & ones(TAG_BITS) }

pub fn with_tag<'a, P, T: 'a>(p: P, tag: usize) -> P
where P: Into<NullablePtr<'a, T>> + TryFrom<NullablePtr<'a, T>> {
    let p = p.into();
    let n = (p.0 & !(ones(TAG_BITS))) | tag;
    P::try_from(NullablePtr::new(n)).unwrap_or_else(|_e| panic!("failed conversion")) }


\end{lstlisting}

%
% \code{ones(k)} returns the bitmask with the \code{k} lower bits set, and \code{TAG_BITS} is a
% predefined number of bits allowed to use for tagging for any pointer. We convert from \code{P} to
% \code{NullablePtr} with \code{.into()} \coderef{TA2}.  In \code{with_tag} we need to use
% \code{TryFrom}, which is a conversion trait that may fail. In CMR \code{Ptr<T>} implements
% \code{TryFrom<NullablePtr>}, where the conversion fails if the \code{NullablePtr} is null. We
% assert that this failure should never happen \coderef{TA9} with the rationale that if we converted
% some type \code{P} into a \code{NullablePtr} and changed its tag, we should be able to convert back
% to \code{P}, even though the conversion is not always possible in general.
%
%
%
%
%
% \subsubsection{Common Operations}
%
%
% \fixme{08/05 15:42 Should talk somewhere about the registering process}
% Note that we do not need consider a reclamation pass happening concurrently in \code{alloc},
% despite there being a window in between the allocation and the pointers protection in the
% \code{Guard}. This is because CMR only frees \emph{registered} addresses. Since we protect the
% pointer in the \code{Guard} \coderef{A3} before registering it \coderef{A4}, the pointer will be
% protected when it is registered, and hence is subject for decallocation.
%
%
%
% \section{Complications}
% \fixme{10/05 18:30 SignalVec?}
% \fixme{10/05 18:30 Initialization stuff?}
% \fixme{something something pitfalls}
% \lorem%
%
% \subsection{Locks in libc\label{sec:alloc-lock}}
%
% In order to protect programmers from deadlocks, POSIX defines a subset of functions as
% \emph{async-signal safe}, meaning they are safe to call from a signal handler. Functions that are
% async-signal safe includes \code{time()}, \code{open()}, and \code{mkdir()}, but it does \emph{not}
% include \code{malloc()}. As such, allocation in signal handlers is not safe, and is a source of
% deadlock bugs. This itself was not a large problem for CMR, as its signal handler did not require
% any allocation. However, as threads are frozen by the reclaimer in a signal handler, it is also not
% safe for the \emph{reclaimer} to call \code{malloc}, despite not being in a signal handler itself.
% This is because some thread may be in the process of allocating memory, and have aquired a lock
% internal to libc, right before being signaled. The thread is still holding the libc lock and is
% frozen in its signal handler by the consolidator, which prevents \emph{all threads}, even those
% oblivious to CMR, from allocating.
%
% This problem is not solved properly by CMR, but its effects are mitigated by wrapping the general
% allocator in Rust to go through yet another lock, the \code{alloc_lock}, which can be aquired by
% the reclamating thread (this is why we have~\coderef{R3} in \cref{lst:reclaimer}). This
% prevents most allocations of deadlocking, but not all. Rust uses \gls{pthreads} internally for thread
% handling on Linux, which allocates internally, both in \code{spawn} and \code{join}. The former may
% be circumvented by aquiring the allocation lock before calling it, but this is no solution for the
% latter, since the thread may depend on allocating before exiting.
% \fixme{deactivate/reactivate around joins?}
%
% \fixme{add example or something here}
%
% \subsection{SignalVec}
%
% \subsection{Thread Registration}
