\chapter{Implementation}

In this chapter we look at the Rust implementation of CMR\@.  We look at four things: how thread
freezing is implemented (\cref{sec:signals}), how reachability analysis is done
(\cref{sec:reachability}), thread and process intercommunication
(\cref{sec:thread-communication}), and how the primitives and operations as defiend in
\cref{ch:cmr} are implemented (\cref{sec:data-types}).


\section{The Reclaimer\label{sec:thread-communication}}

There is only at most one reclaimer at any time. The first thing a potential reclaiming thread
tries to do, is take the lock~\coderef{R2}. Then we take another lock that is used for
allocation~\coderef{R3} (see \cref{sec:alloc-lock}). This may block, but since we have not
frozen any threads yet, any thread holding the lock will eventually release it, assuming it itself
is not deadlocked. Next we initialize shared variables, such as the counters used by threads in
their signal handler. After they are initialized, we freeze all threads~\coderef{R5}, and count how
many threads were successfully frozen, which is used in~\coderef{R6} where we wait on all threads
to finish their job. When done, the data that the threads write out, their \mc{Guard}s and
allocations, are read~\coderef{R7}.

\begin{figure}[ht]
  \begin{lstlisting}[
    numbers=left,
    numberstyle=\color{gray}\ttfamily{}R,
    caption=Pseudocode of the work of the reclaimer,
    label=lst:reclaimer,
  ]
run_reclaim_pass() {
    if reclaim_lock.lock().fail() { return }
    lock_malloc()
    init_shared_vars()
    n = freeze_threads()
    while sh_done_counter.load() != n { wait() }
    (guards, allocs) = read_thread_datas()
    memory_map = mmap()
    write_marker(memory_map)
    if fork() == Child {
        rs = find_reachable()
        write_to_mmap(rs, memory_map)
        exit(0) }
    sh_frozen.store(false)
    send_to_background_thread(allocs, memory_map) }\end{lstlisting}
\end{figure}

When \code{fork()}ing the child process continues the thread of the parent process that called
\code{fork()}, such that it has access to everything that the original thread had\fixme{rewrite}.
As such, we don't need to communicate from parent to child. However, the job of the child process
is to run reachability analysis, and we do need its result. CMR uses \emph{memory map}
(\cref{sec:memory-map}) for \gls{ipc}\@, which is set up before the fork. Since we need to
know when the child is done writing its results, we write a \emph{marker} word as the first word in
the memory map. Then we \code{fork()}. The child process does the reachability analysis, and
writes the result after marker in the memory map. When it is done writing, it overwrites the marker
with the number of elements written.

\begin{figure}[b]
  \centering
  \begin{tikzpicture}[node distance=0.7cm]
  \tikzstyle{mem} = [draw, fill=white, outer sep=0mm, minimum width=1cm,
                     minimum height=0.5cm, node distance=0.5cm];
  \tikzstyle{parent} = [fill=white,draw,minimum width=3.5cm];
  \tikzstyle{child} = [fill=white,draw,minimum width=2.5cm];
  \tikzstyle{padbox} = [draw,fill=gray!15, inner sep=2mm];

  \node [parent,fill=none,draw=none] (parent-label) {Parent Process};
  \node [parent] (t1) [below of=parent-label] {$T\sb{1}$};
  \node [parent] (t2) [below of=t1]           {$T\sb{2}$};
  \node [parent] (t3) [below of=t2]           {$T\sb{3}$};
  \node [parent] (bt) [below of=t3]           {Background Thread};

  \node [child,fill=none,draw=none,node distance=7cm] (child-label) [right of=parent-label] 
      {Child Process};
  \node [child]             (t'2)         [below of=child-label]  {$T'\sb{2}$};

  \node (mmap-label) at ($(parent-label.east)!0.5!(child-label.west)$) {\texttt{mmap}};
  \node [mem] (m1) [below of=mmap-label] {\texttt{0xbeef}};
  \node [mem] (m2) [below of=m1]         {\texttt{0xcafe}};
  \node [mem] (m3) [below of=m2]         {\texttt{0xbabe}};
  \node [mem] (m4) [below of=m3]         {\texttt{0xdead}};
  \node [mem] (m5) [below of=m4]         {\texttt{0xabba}};

  \begin{scope}[on background layer]
    \node [padbox,fit={(parent-label) (bt)}] {};
    \node [padbox,fit={(child-label) (t'2)}] {};
    \node [draw,fit={(mmap-label) (m5)},inner sep=0mm,fill=gray!15] {};
  \end{scope}

  \draw[-latex] (bt.east) -| ($(bt.east)!0.5!(m1.west)$) |- (m1.west);
  \draw[-latex] (t'2.west) -| ($(t'2.west)!0.5!(m4.east)$) |- (m4.east);
\end{tikzpicture}

  \caption{Illustration of \gls{ipc} through a memory map. $T\sb{2}$ in the parent process is the
  reclaiming thread, so $T\sb{2}$ is the one thread in the child process. Both processes have
  access to the same memory in the memory map.}
\end{figure}

In order to minimize the delay of the reclamation pass from the point of view of the user, we spawn
a background thread which handles the part of the reclamation performed after we \code{fork()}.
This ensures that no user thread has to wait for the child process to finish its reachability
analysis before going back to running application code.  After forking, the parent process
unfreezes the other threads, and sends all necessary data to the background thread. Note that the
\code{reclaim_lock} is not released, even though the thread is exiting the procedure. The lock is
only released when the background thread is fully done with the reclamation pass.

\section{Thread Freezing\label{sec:signals}}

\emph{Signals} is a process communication mechanism used by POSIX compliant operating systems.
\gls{pthreads} also supports signaling for communication between threads in a
process. We utilize this in order to implement thread freezing, by registering a signal handler for
the signal \code{SIGUSR1}. Pseudocode for the signal handler CMR uses is shown in
\cref{lst:guard-impl}. The signal handler is registered with the \code{sigaction}
function, and threads are signaled with \code{pthread_sigqueue}.

\begin{figure}[b]
  \begin{lstlisting}[
    numbers=left,
    numberstyle=\color{gray}\ttfamily{}SH,
    caption=Pseudocode for the signal handler used by CMR,
    label=lst:signal-handler,
  ]
if is_in_cleanup:
    clean_up_and_return()
id = sh_enter_counter.fetch_add(1)
write_out_data_to(thread_datas[id])
sh_done_counter.fetch_add(1)
while sh_frozen.load():
    wait()
sh_enter_counter.fetch_sub(1)\end{lstlisting}
\end{figure}

Due to a complication in which a threads shutdown procedure is initiated without the signal handler
being removed, we need to check if we are shutting down~\coderef{SH1}, as we do not have access to
the thread local storage, which we need. If we are shutting down we will not participate in the
rest of the handler, but clean up in order to avoid deadlocks~\coderef{SH2}.
We use \code{sh_enter_counter} to keep track of how many threads are present in the signal handler;
the reclaimer knows how many threads it succesfully signaled, so it knows how many threads to
expect. \coderef{SH3} registers a threads presence, in addition to giving each thread a unique
index in the range $[0, n)$, where $n$ is the number of threads signaled. This is used
in~\coderef{SH4}, where each thread writes out their guards and allocations into the global vector
\code{thread_datas}. We then register that we have written our data~\coderef{SH5}, and wait for the
reclaimer to unfreeze us~\coderef{SH6}.
\fixme{Since we don't know that we have exited the handler after SH6, do we really need SH7??}



The signal handler is registered by a thread with a call to the \code{thread_activate} procedure,
which must be called before the thread is using CMR\@. \fixme{bench automation of this in Guard or
something?}
The system has a fixed upper bound on the number of threads that may be active at any time, through
the \code{thread_datas} vector, which is pre-allocated with a fixed size \fixme{check this out}.

Since a thread handler may have registered it leaving by decrementing \\\code{sh_enter_counter},
but then preempted before leaving the handler, CMR uses \\\code{pthread_sigqueue} instead of the more
standard \code{pthread_kill}, as \gls{pthreads} may simply ignore a signal that is sent to a
thread when that thread is already in the signal handler for a given signal.
This is problematic, since the reclaimer keeps carefully track of where threads are in their
handler, and waits for all threads to reach certain checkpoints. If the reclaimer is lead to
believe that a thread just entered the handler when in fact it was just leaving, it will lead to a
deadlock.


\section{Reachability\label{sec:reachability}}

Reachability analysis is done through a standard mark-and-sweep algorithm. The reclaimer collects
all roots from all active threads, and looks through the memory of each data type that is pointer
to by the roots. If any data type contains a \mc{Guard} or \mc{Ptr}, the address is marked as seen,
and added to the frontier. When the frontier is empty, we have registered all reachable memory.
\cref{lst:find-reachable} shows the algorithm, which is effectively breadth-first search
through the memory graph.

\begin{figure}[ht]
\begin{lstlisting}[
  caption=The algorithm used to find all reachable memory blocks.,
  label=lst:find-reachable,
  frame=None,
  basicstyle=\normalsize\mdseries,
  columns=flexible,
  xleftmargin=15pt,
  numbers=left,
  keywordstyle={\bf},
  mathescape,
  commentstyle={\color{gray}},
  keywords={for,if,else,return,in,to,while,continue,break,output,not}]
$\textsc{find-reachable(Roots)}$
    Frontier = Roots
    Seen = Roots
    while mem = Pop(Frontier)
        for ptr in Ptrs(mem)
            if ptr not in Seen
                Push(Frontier, ptr)
            Insert(Seen, ptr)
    return Seen
\end{lstlisting}
\end{figure}

Finding pointers in arbitrary data types might involve significant work since the size of the data
types can be arbitrarily large, in addition to that memory might not be initialized, and false
positives. Instead of scanning through the memory block linearily, \fixme{should write somewhere why
we can assume that the ptr is the start of a block} CMR has a Trait called \code{Trace}, which all
data types that is in shared memory must implement.  A type implementing \code{Trace} knows a bound
on how many shared memory pointers it contains, and can write these out to a buffer. For instance,
a \code{Node} in a single linked list contains only one pointer, namely its next pointer, which is
trivial to write out.

The implementation of this uses \emph{Trait Objects} (\cref{sec:trait-objects}), which involves
dynamic dispatch. This solution is potentially expensive, as it may involve cahce misses in the
I-cache, although the number of misses is limited by the difference in data types in shared memory,
which normally is smaller than in Rust memory.

\begin{figure}[ht]
  \begin{lstlisting}[
  style=Rust,
  caption=Definition of the \code{Trace} trait and a sample implementation for a
  linked list node. The impementation uses \emph{specialization} (\cref{sec:specialization})
  as the implementation of \code{Node}s containing data that itself is \code{Trace} is different.,
  numbers=left,
  numberstyle=\color{gray}\ttfamily{}T,
  ]
pub trait Trace { fn count(&self) -> usize { 0 }
                  fn write(&self, &mut [TraitObject]) -> usize { 0 } }
pub struct Node<T> { data: ManuallyDrop<T>,
                     next: Atomic<Node<T>> }
impl<T> cmr::Trace for Node<T> {
    default fn count(&self) -> usize { 1 }
    default fn write(&self, slice: &mut [TraitObject]) -> usize {
        let p = unsafe { self.next.load(SeqCst) };
        if !p.is_null() { slice[0] = ptr::trait_object(p);
                          1 } 
        else { 0 } } }\end{lstlisting}

\end{figure}

\code{Trace} contains default implementations of the two functions, such that primitive types can
easily implement it. \code{write} takes a buffer, writes all pointers to it as \code{TraitObject}s,
and returns the number of objects written. \code{count} gives an upper bound on the number of
pointers written. This is useful for collection types, like \code{Vec} or \code{HashMap}, which
also may contain pointers to shared memory.

\code{Node} is a standard node from a linked list, containing \code{data}, and a \code{next}
pointer. The implementation of \code{write} loads the \code{next} pointer~\coderef{T12}, which is
an \code{unsafe} operation, as there is no \mc{Guard} protecting the pointer. This is safe in the
context of the reclaimer since the memory will be freed at earliest when we finish the reachability
analysis, and at that point we no longer read the memory. The implementation only writes out the
pointer if it is non-null. While this is not required for CMR to function, it simplifies the logic
in the reachability analysis.


\section{Data Types\label{sec:data-types}}

In this section we look at the concrete Rust implementation of selected data types and operations,
and argue for their correctness.


\subsubsection{Guard}

The \mc{Guard} is implemented as a single word, in addition to an empty type (the
\code{PhantomData}) as Rust \emph{requires} generic types to be used. \mc{Guard}s aren't normally
constructed directly (\fixme{non movable types somewhere}), but rather declared with the
\code{guard!} macro, which constructs it and calls \code{Guard::register}.
An excerpt of the definitions of \mc{Guard} is shown in \cref{lst:guard-impl}.

\begin{lstlisting}[style=Rust,
caption=Excerpt of \mc{Guard}s definitions,label=lst:guard-impl]
struct Guard<T> {
  ptr: usize,
  _marker: PhantomData<T>,
}
impl<T: Trace> Guard<T> {
    pub unsafe fn new() -> Self { Guard { inner: 0, _marker: PhantomData, } }
    pub fn copy_guard(&mut self, other: &Self) { self.inner = other.inner; }
    pub fn register(&mut self) {
        ROOTS.with(|r| { let mut v = r.borrow_mut();
                         v.push(GuardPointer::from_guard(self)) });
    }
    ...
}
macro_rules! guard {
    ($var:ident) => { let $var = unsafe { &mut $crate::guard::Guard::new() };
                      $var.register();
    }
}\end{lstlisting}


\fixme{rewrite, focus on \code{register}}

Since we need to keep track of type information dynamically, we construct a \code{GuardPointer}
which does exactly this, and pushes it onto a thread local \code{Vec}, \code{ROOTS}.  It is also
possible to construct a \mc{NullablePtr} from the \mc{Guard}.  The \code{guard!} macro corresponds
to make-guard (\cref{eq:make-guard}). copy-guard (\cref{eq:copy-guard}) is shown in the Listing.

Using a macro for declaring \mc{Guard}s is a trick that serves two purposes:
\begin{enumerate*}[1) ]
  \item the user needs to write \code{unsafe} in order to make a \mc{Guard} without calling
    \code{register} on it,
  and\item the user never gets a binding to a variable of type \code{Guard}, but only of type
    \code{\&mut Guard}, which prevents moving the \mc{Guard}, hence fulfulling the requirement of a
    \mc{Guard} being non-movable from \cref{def:guard}.
\end{enumerate*}

In addition to the \code{guard!} macro for declaring new \mc{Guard}s, CMR also defines the
\code{guard} \emph{procedure}, which loads an \mc{Atomic} into a \mc{Guard}.


\subsubsection{Atomic}

\mc{Atomic} is mainly a wrapper around Rusts \code{AtomicPtr}, although the internals differ
slightly. CMR defines its own type so that we can defide on the return types of certain functions.
We include the definition of the \code{struct}, as well as \code{cas}, the compare-and-swap
operation, in which we utilize some Traits from the Rust standard library to convert between types.
\fixme{write a little here}

\begin{lstlisting}[caption=Excerpt of \mc{Atomic}s definitions (Trait bounds omitted for brevity)]
pub struct Atomic<T> {
    data: AtomicUsize,
    _marker: PhantomData<T>,
}
impl<T> Atomic<T> {
    pub fn cas<'a, A, B>(&self, a: A, b: B, ordering: Ordering)
     -> Result<A, NullablePtr<'a, T>> {
        let (old, new£ = (raw(a£, raw(b££;
        let ret = self.data.compare_and_swap(old, new, ordering£;
        if ret == old { Ok(A::try_from(NullablePtr::new(ret££.unwrap(££ }
        else { Err(NullablePtr::new(ret££ }
    }
    ...
}

\end{lstlisting}



\subsubsection{NullablePtr}

\code{NullablePtr} is used as the canonical pointer type in CMR, and all pointer like types are
converted to \code{NullablePtr} using the \code{From} \code{Into} traits, which
handles coversion between types. For instance, we implement \code{From<*const T> for
NullablePtr<T>}. This way we can write functions that are generic over all types of pointers, so
that the user of CMR does not have to handle these conversions themselves. This is used in the
functions for pointer tagging, as well as the \code{cas} in \cref{lst:atomic} (the types \code{A}
and \code{B}). \cref{lst:tagging} shows some of the free functions for pointer tags that are
generic over different pointer types.

\begin{lstlisting}[
  numbers=left, numberstyle=\color{gray}\ttfamily{}TA,
  caption=Implementation of pointer tagging functions,
  label=lst:tagging]
pub fn tag<'a, P, T: 'a>(p: P) -> usize where P: Into<NullablePtr<'a, T>> {
    let n: NullablePtr<T> = p.into();
    n.0 & ones(TAG_BITS) }

pub fn with_tag<'a, P, T: 'a>(p: P, tag: usize) -> P
where P: Into<NullablePtr<'a, T>> + TryFrom<NullablePtr<'a, T>> {
    let p = p.into();
    let n = (p.0 & !(ones(TAG_BITS))) | tag;
    P::try_from(NullablePtr::new(n)).unwrap_or_else(|_e| panic!("failed conversion")) }


\end{lstlisting}


\code{ones(k)} returns the bitmask with the \code{k} lower bits set, and \code{TAG_BITS} is a
predefined number of bits allowed to use for tagging for any pointer. We convert from \code{P} to
\code{NullablePtr} with \code{.into()} \coderef{TA2}.  In \code{with_tag} we need to use
\code{TryFrom}, which is a conversion trait that may fail. In CMR \code{Ptr<T>} implements
\code{TryFrom<NullablePtr>}, where the conversion fails if the \code{NullablePtr} is null. We
assert that this failure should never happen \coderef{TA9} with the rationale that if we converted
some type \code{P} into a \code{NullablePtr} and changed its tag, we should be able to convert back
to \code{P}, even though the conversion is not always possible in general.





\subsubsection{Common Operations}

In addition to the types member function, we have defined two important free functions:
\code{guard} and \code{alloc}.
\code{guard} takes a \code{\&mut Guard} and \code{\&Atomic}, \code{load}s the atomic, and protects
the pointer read in the \code{Guard}, without a reclamation pass happening in between --- this is
ensured by the \code{atomic} call.

\begin{lstlisting}
pub fn guard<'a, T>(guard: &'a mut Guard<T>, a: &Atomic<T>) -> NullablePtr<'a, T> {
    atomic(|| { let p = unsafe { a.load(SeqCst) };
                guard.inner = ptr::raw(p);
                p }) }
\end{lstlisting}
\code{alloc} is similar to \code{Box::new}, as it just wraps \code{cmr::alloc::alloc} in addition
to protecting the pointer returned in the \code{Guard} passed in.
\begin{lstlisting}[numbers=left, numberstyle=\color{gray}\ttfamily{}A]
pub fn alloc<T: Trace>(guard: &mut Guard<T>, t: T) -> Ptr<T> {
    let ptr = alloc::alloc(t);
    guard.inner = ptr::addr(ptr);
    alloc::register(ptr);
    ptr }
\end{lstlisting}

\fixme{08/05 15:42 Should talk somewhere about the registering process}
Note that we do not need consider a reclamation pass happening concurrently in \code{alloc},
despite there being a window in between the allocation and the pointers protection in the
\code{Guard}. This is because CMR only frees \emph{registered} addresses. Since we protect the
pointer in the \code{Guard} \coderef{A3} before registering it \coderef{A4}, the pointer will be
protected when it is registered, and hence is subject for decallocation.



\section{Complications}
\lorem%

\fixme{something something pitfalls}

\subsection{Locks in libc\label{sec:alloc-lock}}

In order to protect programmers from deadlocks, POSIX defines a subset of functions as
\emph{async-signal safe}, meaning they are safe to call from a signal handler. Functions that are
async-signal safe includes \code{time()}, \code{open()}, and \code{mkdir()}, but it does \emph{not}
include \code{malloc()}. As such, allocation in signal handlers is not safe, and is a source of
deadlock bugs. This itself was not a large problem for CMR, as its signal handler did not require
any allocation. However, as threads are frozen by the reclaimer in a signal handler, it is also not
safe for the \emph{reclaimer} to call \code{malloc}, despite not being in a signal handler itself.
This is because some thread may be in the process of allocating memory, and have aquired a lock
internal to libc, right before being signaled. The thread is still holding the libc lock and is
frozen in its signal handler by the consolidator, which prevents \emph{all threads}, even those
oblivious to CMR, from allocating.

This problem is not solved properly by CMR, but its effects are mitigated by wrapping the general
allocator in Rust to go through yet another lock, the \code{alloc_lock}, which can be aquired by
the reclamating thread (this is why we have~\coderef{R3} in \cref{lst:reclaimer}). This
prevents most allocations of deadlocking, but not all. Rust uses \gls{pthreads} internally for thread
handling on Linux, which allocates internally, both in \code{spawn} and \code{join}. The former may
be circumvented by aquiring the allocation lock before calling it, but this is no solution for the
latter, since the thread may depend on allocating before exiting.
\fixme{deactivate/reactivate around joins?}

\fixme{add example or something here}



% \clearpage
%  \vspace{2cm}
%
%
%  The key idea of the implementation is to use an operation known as \emph{forking} to obtain a
%  snapshot of the entire memory space of the process. Having this we can identify the roots, and run
%  reachability analysis, knowing that nothing will change while we are looking.
%
%  Consolidation starts by sending all running threads a \emph{signal}, causing them to jump to a
%  \emph{signal handler}. The signal handler registers that the thread has been signaled, such that the
%  garbage collection pass is detectable. Then the thread writes out the address of a thread local
%  list of allocated addresses, such that the consolidator can collect all addresses that has been
%  allocated since the last consolidation. At last, the threads wait for the consolidator to say that
%  they may return to their execution. \fixme{why is this required?}
%
%  \fixme{Add implementation details here}
%
%    \code{WAS\_SIGNALED} is a thread local flag that is
%    set to \code{true} in the signal handler. Note that this function does not support being called
%    recursively.
%  \begin{figure}[ht]
%    \begin{lstlisting}[numbers=left,numberstyle=\color{gray}\ttfamily{}A,caption=Pseudocode of $atomic$]
%  fn atomic(f: F):
%    loop:
%      WAS_SIGNALED = false
%      let ret = f()
%      if not WAS_SIGNALED: return ret
%  \end{lstlisting}
%  \end{figure}
%
%  \begin{theorem}
%    If \code{f} does not itself call \code{atomic} then \code{atomic} satisfy
%    \cref{eq:atomic}.
%  \end{theorem}
%  \begin{proof}
%    \code{WAS\_SIGNALED} is set to \code{false} in \code{A3}, so if it is observed to be
%    \code{true} in \code{A5} the signal handler must have been executed. Since only the consolidator
%    signals any thread, it means that there was a consolidator in this time interval. We are only
%    returning from the function if \code{WAS\_SIGNALED = false}, so if we return \code{f} was
%    successfully called without overlapping with a pass.
%  \end{proof}
%  Note that it is possible that the consolidator was initiated before entering the \code{atomic}
%  function, but that they only got execution time after \code{A3}.
%  \begin{lemma}
%    If the consolidator is lock-free, then \code{atomic} is lock-free.
%  \end{lemma}
%  \begin{proof}
%    If we loop, there is a consolidator. Since the consolidate process has a finite number of steps
%    and is lock-free, it will either make progress, or some other thread is making progress. In
%    either case, the system makes progress.
%    \fixme{this is a terrible proof}
%  \end{proof}
%
%  It is important that the function given to \code{atomic} can be called multiple times without
%  breaking. The most useful case for \code{atomic} is a \code{load/store} store pair, so this is
%  usually the case.
