\chapter{Background}


\section{Garbage Collectors}

\fixme{31/05 10:58 What types of GC are there?}
A \emph{Garbage Collector} usually refers to an automatic subsystem that handles memory management
without requiring programmer assistence. Many widespread language implementations,
including Java, Python, and Go, use a garbage collector, although the internal details of each
system varies greatly.

\fixme{31/05 10:58 More here}
The job of the garbage collector is to identify memory segments that are no longer used by the
program. One way of doing this is to represent the program memory as a graph $G=(V, E)$ where $V$ is
all allocated memory segments and $(u, v) \in E$ if the region $u$ contains an address that is
inside the segment $v$. One consequence of this model is that memory addresses cannot be computed
from other values.

\fixme{31/05 10:58 Discussion about GC in general?}


\section{Operating Systems}
\lorem{}

\subsection{Virtual Memory}
\lorem{}

\subsubsection{Memory Maps\label{sec:memory-map}}
\lorem{}


\subsection{Signals}
\lorem{}


\subsection{Threads and Processes}
\lorem{}


\section{Programming Languages}
\lorem{}


\section{Concurrency}
\lorem{}

\subsection{Common Patterns in Concurrent Programming\label{sec:common-patterns}}
\lorem{}

\subsection{The ABA-Problem\label{sec:aba-problem}}


\section{Memory Reclamation}
\lorem{}

\subsection{Hazard Pointers\label{sec:hazard-pointers}}
\lorem{}

\subsection{Forkscan\label{sec:forkscan}}
\lorem{}


\section{Related Works}
\lorem{}

\subsection{Crossbeam}
\lorem{}
