\chapter{Rust\label{ch:rust}}


\section{Introduction}
\lorem{}



\section{The Borrow Checker\label{sec:borrow-checker}}

A central concept in Rust is that of ownership. At any moment, an object has exactly one binding
which \emph{owns} the object. Ownership may be transfered (``\emph{moved}'', which is the default
behaviour), or it may be borrowed, of which there are two types: immutable borrows and mutable
borrows. The borrowed binding is in effect a \emph{reference} to the data, similiar to references
in other programming languages. The three types of ownership handling is shown in
\cref{fig:rust-ownership}.

\begin{figure}[ht]
  \begin{subfigure}{0.3\textwidth}
\caption{Ownership transfer}
    \begin{lstlisting}
fn bar(f: Foo);
let x = foo();
bar(x);\end{lstlisting}
  \end{subfigure}%
  \hfill
  \begin{subfigure}{0.3\textwidth}
\caption{Immutable Borrow}
    \begin{lstlisting}
fn bar(f: &Foo);
let x = foo();
bar(&x);\end{lstlisting}
  \end{subfigure}%
  \hfill
  \begin{subfigure}{0.3\textwidth}
\caption{Mutable Borrow}
    \begin{lstlisting}
fn bar(f: &mut Foo);
let mut x = foo();
bar(&mut x);\end{lstlisting}
  \end{subfigure}%
  \caption{The three types of ownership handling.\label{fig:rust-ownership}}
\end{figure}


\fixme{11/05 14:05 what happens after a value has been moved? Ex with \code{Vec}.}
One of the reasons to differentiate between mutable and immutable borrows, is references in Rust
can be either aliased, or mutable. That is, if there is a mutable reference to some object, then
that reference has to be the \emph{only} reference. This ensures that immutable references are
never changed.



\section{Lifetimes\label{sec:rust-lifetimes}}

Lifetimes is the second important concept in Rust. The idea of lifetimes i to have a way of talking
about the scope of a variable --- its lifetime. By tracking the lifetime of all variables at
compile time the Rust compiler is able to catch errors such as returning function local variable
addresses. \cref{lst:lifetimes} shows an example function attempting to do this.
\begin{lstlisting}[style=Rust,label=lst:lifetimes]
fn foo(_a: &i32) -> &i32 {
  let num: i32 = 420;
  let r: &i32 = &num;
  r }
\end{lstlisting}

Since Rust tracks the lifetime of all variables, it knows that the lifetime of \code{num} is the
same as that of the function body. The lifetime of \code{r} is the same, as it is a reference to
\code{num}. So when we try to return \code{r} in the last line of the function, Rust realizes that
the lifetime of the reference we return ends its life at the end of the function; this is clearly
not what we wanted, since it would make the returned reference dead on arrival. Compilation fails
with the following error: \code{error[E0597]: `num` does not live long enough}.

\code{struct}s can also be annotated with lifetimes, and in fact is required to be so if any of its
members are references. This is because the lifetime of the struct is bounded by the lifetime of
its member variables.
\begin{lstlisting}
struct<'a> {
  age: i32,
  name: &'a str }
\end{lstlisting}

Although Rust programmers may have to think about the lifetime of the variables, they seldom have
to write lifetime annotated functions, due to \emph{lifetime elison} --- the compiler can ususally
figure out the most general lifetime that fits the function. For instance, \code{foo(\&str) ->
\&str} is considered by the compiler as \code{foo<'a>(\&'a str) -> \&'a str}.

\fixme{11/05 14:04 example of the ptr struct here showing lifetime and why its useful for us?}



\section{Unsafe Rust}

When talking about the Rust progrmaming language, one usually talks about a subset of Rust, called
\emph{Safe Rust}. In Safe Rust, there are no race conditions, mutable memory locations are never
aliased, and all pointer accesses are valid.  The real world, on the other hand, offers seldom
these guarantees, and the unfortunate truth which Rust programmers must deal with is that in order
to implement some of these safe abstractions we want (like \code{Vec}, \code{Mutex}, and
\code{Box}), some unsafety is required.  For this reason, Rust offers an escape hatch for some of
its rules: \emph{Unsafe Rust}.

The difference between Safe and Unsafe Rust is only four things. In Unsafe Rust one may:
\begin{enumerate*}[1) ]
    \item dereference raw pointers
    \item mutate statics.
    \item call \code{unsafe} functions
    \item implement \code{unsafe} traits
\end{enumerate*}

Dereferencing raw pointers is naturally \code{unsafe}, as it is not possible to statically
guarantee that the address of the pointer is valid memory, or that the objects it points to is
still alive. Mutation of \code{static}  variables is also unsafe due to the lack of thread
synchronization.

\code{unsafe} functions and traits are just a marker added to the function or trait, signaling that
not all uses of this is guaranteed to be safe. For instance, the trait \code{Send} is a marker
trait, and types implementing \code{Send} may be sent across thread boundaries. While this is fine
for most types, there are types which does not allow this. The reference counted pointer
\code{Rc<T>} is an example, which is a pointer to a tuple\footnote{Not really, but for our purposes
here we can pretend that it is.} \code{(count, data)}. The \code{count} is incremented each time
\code{.clone()} is called, and decremented when a variable is \code{Drop}ped.  To understand why
this cannot be send across thread boundaries safely, consider what happens if $T\sb{1}$
\code{.clone()} at the same time as $T\sb{2}$ \code{Drop}s it: the \code{count} field is written to
twice without any synchronizationor atomic operations\footnote{\code{Rc} does not use atomics for
performance reasons, but \code{Arc} does, and it does implement \code{Send}.} --- a race condition!

One way of thinking about the unsafety of ones codebase is that there should be no undefined
behaviour in safe code, no matter how the code looks like. In other words, it should be impossible
to mess up so badly as to invoke undefiend behaviour without typing \code{unsafe}.



\section{Concurrency}

bing bong

\subsection{Concurrency and Aliasing}

One observation to make from the reference rules as presented in \cref{sec:borrow-checker} is that
since references are either aliased or mutable, then there can be no writes shared data between
threads, in Safe Rust, even using atomics. While this is \emph{technically} true, the Rust standard
library uses \code{\&T} and \code{\&mut T} slightly different than ``immutable'' vs ``mutable'' in
this context: \code{\&T} means that the type may be shared between threads.

Take \code{AtomicUsize} as an example, a \code{usize} exposing atomic operations like \code{store},
\code{load}, and \code{compare_and_swap}, which signatures are shown in \cref{lst:atomicusize}.
\begin{lstlisting}[label=lst:atomicusize,caption=Signatures for selected operations on
\code{AtomicUsize}]
pub fn load(&self, order: Ordering) -> usize;
pub fn store(&self, val: usize, order: Ordering);
pub fn swap(&self, val: usize, order: Ordering) -> usize;
pub fn compare_and_swap(&self, current: usize, new: usize, order: Ordering) -> usize;
\end{lstlisting}

Clearly, \code{AtomicUsize::store} modifies memory of the \code{usize}; despite this the function
is \code{\&self} and not \code{\&mut self}, since the operation is allowed on variables which are
shared between threads.
This is a useful distinction, since we can have methods on \code{AtomicUsize} that \emph{is}
\code{\&mut self}, which then is only possible to invoke should the variable not have been shared
between threads yet. For instance, \code{AtomicUsize::get_mut(\&mut self) -> \&mut usize} allows
the underlying \code{usize} to be changed without any synchronization overhead.

\fixme{09/05 11:48 notes on common patterns. MP, Rayon?}
\subsection{The Standard Library}

The standard librarys synchronization module \code{std::sync} contains primitives that most
concurrent programs require, such as \code{Mutex}, \code{Channels}, \code{Condvar}, and
\code{Atomic}s.

\subsection{Common Patterns}

It is common among Rust programmers to build abstractions over lower level primitives. For
instance, a common pattern in parallel and concurrent programming is to have a \emph{thread pool},
which is given work, and internally handles the thread synchronization and work division. Example
usage of such an abstraction could be \code{let tp = ThreadPool::new(); tp.execute(|| { ... });}.

Another example is data parallelism: given some collection of data we want to iterate over the
elements and perform some operation on each element. The Rust library \code{rayon} offers exactly
this: parallel iterator. Instead of writing \code{vec.iter()} to iterate over a \code{Vec}, with
\code{rayon} we can write \code{vec.par_iter()}, and get data parallelism for free. Internally
\code{rayon} uses a thread pool and work stealing to handle the division of labour among the
threads.


\fixme{09/05 12:32 Look at stuff from last semester. Move into std sec?}
\subsection{Memory Orderings}

\section{Nightly Rust}

The Rust language and compiler follows a fixed release schedule, where a new stable version is
released every six weeks. In addition to this there is the beta branch, which is the upcomming
version, and the nightly version which is the most recent version, build daily from the
\code{master} branch of the source tree.

The nightly version of the compiler allows users to opt in on \emph{untsable} features: features
that are partially or fully implemented, but which details are not yet committed to. These featuers
includes new APIs in the standard library, new syntax, and new language features all together.
As we have used multiple unstable features in CMR, we look at some of them in deatil.


\subsection{Non-Lexical Lifetimes\label{sec:nll}}
The current implementation of lifetime checking in the compiler is \emph{lexical}, meaning
variables are live until they go out of scope, despite not being used. This is a limitation that
one may want to get rid off. The feature \gls{nll} lifts this requirement, and lets the lifetime of
a variable last only until its last usage. Having this it is possible to seemingly break some of
Rust rules, like aliased mutable references:
\begin{lstlisting}[style=Rust]
let mut v = vec![1,2,3];
let r1 = &mut v;
let r2 = &mut v;
\end{lstlisting}

This will compile, as we do not use \code{r1} after having made \code{r2}. If we write
\code{r1.push(1);} after \code{let r2}, we get the same error as without using \gls{nll}.



\subsection{Trait Objects\label{sec:trait-objects}}

When using traits in function signatures or structs we can either make the struct generic over some
type that implements the trait, or we can use dynamic dispatch. As generics usually are implemented
with copying the source for each invocation of a new type, it increases code size and compilation
time. In addition, collections and similar structures cannot mix different types: a \code{Vec<Tr>}
cannot both contain elements of type \code{A} and \code{B}, even if both implements \code{T}.

Dynamic dispatch is the other option. Now variables are \emph{fat pointers}, containing both the
pointer to the data type, and a pointer to a \fixme{10/05 13:29 check
this}\code{vtable}\footnote{the name \code{vtable} comes from the C++ world, where function on
abstract types are called \code{virtual} functions}, which contains information about the function
addresses for that type, as shown in \cref{fig:trait-objects}. The entry in the \code{vtable} is
all functions for some trait. With this we can take any concrete type, and follow its vtable
pointer, in order to find the implementation of some trait function for that type. In
\cref{fig:trait-objects}, both \code{Foo} and \code{Bar} implements some trait which have a
function named \code{fnc}. By following the pointers from the stack, we get the data (left) and the
function pointer (right).

\begin{figure}[ht]
  \centering
  \begin{tikzpicture}[node distance=1cm]
\tikzset{split/.style={draw, rectangle split, rectangle split horizontal, rectangle split parts=2,
  outer sep=0mm, node distance=6mm, minimum height=6mm, text width=6mm}}
  \tikzset{full/.style={draw, outer sep=0mm, node distance=6mm, minimum height=6mm, text width=20mm}}
  \tikzset{empty/.style={outer sep=0mm, minimum height=6mm, node distance=6mm}}
  \tikzset{border/.style={draw, inner sep=0mm}};

  % stack block
  \node [] (stack-label) {stack};
  \node [empty] (e1)  [below of=stack-label] {...};
  \node [split] (st1) [below of=e1]  {\null\nodepart{second}\null};
  \node [empty] (e2)  [below of=st1] {...};
  \node [split] (st2) [below of=e2]  {\null\nodepart{second}\null};
  \node [empty] (e3)  [below of=st2] {...};
  \node [border, fit={(e1) (st1) (e3)}] {};


  % vtable block
  \node [node distance=4cm] (vtable-label) [right of=stack-label] {vtables};
  \node [split] (v1) [below of=vtable-label] {\code{fnc}\nodepart{second}\null};
  \node [empty] (ve1) [below of=v1] {...};
  \node [split] (v2) [below of=ve1]           {\code{fnc}\nodepart{second}\null};
  \node [empty] (v-rest) [below of=v2] {...};
  \node [border, fit={(v1) (v-rest)}] {};


  % heap block
  \node [node distance=4cm] (heap-label) [left of=stack-label] {memory};
  \node [full] (h1) [below of=heap-label] {\footnotesize\code{Foo \{x: usize\}}};
  \node [empty] (he1) [below of=h1] {...};
  \node [full] (h2) [below of=he1]         {\footnotesize\code{Bar \{y: \&str\}}};
  \node [empty] (h-rest) [below of=h2] {...};
  \node [border, fit={(h1) (h-rest)}] {};


  % Pointers to vtable
  \draw [circlebase, -latex] ($(st1.center)!0.5!(st1.east)$) -| ($(st1.east)!0.33!(v1.west)$) |- (v1.west);
  \draw [circlebase, -latex] ($(st2.center)!0.5!(st2.east)$) -| ($(st2.east)!0.66!(v2.west)$) |- (v2.west);

  % vtable pointers
  \draw [circlebase, -latex] ($(v1.center)!0.5!(v1.east)$) -- ($(v1.east) + (1, 0)$);
  \draw [circlebase, -latex] ($(v2.center)!0.5!(v2.east)$) -- ($(v2.east) + (1, 0)$);

  % Pointers to data
  \draw [circlebase, -latex] ($(st1.center)!0.5!(st1.west)$) -| ($(st1.west)!0.33!(h1.east)$) |-
  (h1.east);
  \draw [circlebase, -latex] ($(st2.center)!0.5!(st2.west)$) -| ($(st2.west)!0.66!(h2.east)$) |-
  (h2.east);
\end{tikzpicture}

  \caption{Illustration of memory when using Trait Objects.\label{fig:trait-objects}}
\end{figure}

While trait objects offers greater flexibility in the usage of traits, the pointer jumping may lead
to worse cache behaviour, and important compiler optimizations like inlining is impossible.



\subsection{Specialization\label{sec:specialization}}

Specialization is a feature which allows multiple implementations of a trait for the same type,
where the implementations are ordered by their specificity. Assume we have a trait \code{T} for a
struct that is generic over a type \code{S<G>}. Then we can implement \code{impl T for S<G>},
\emph{and} \code{impl T for S<G> where G: O}, where \code{O} is another trait. Now for all types
\code{G}, if \code{G} implements \code{O}, we will use the latter implementation of \code{T}, but
if it does not, we use the former. This require that we mark the functions implemented in the
implementation without any bounds with \code{default}, as shown in \cref{lst:specialization}. Now
\code{Ptr<String>} will use \code{foo} from \coderef{S5} as \code{String} implements
\code{Debug}, while \code{Ptr<SomeStruct>} may use the \code{foo} from \coderef{S2}.

\begin{lstlisting}[
  style=Rust,
  label=lst:specialization,
  caption=Using specialization to implement a trait twice.,]
impl<T> Trait for Struct<T> {
  default fn fmt(&self);
}
impl<T: Debug> Trait for Struct<T> {
  fn fmt(&self);
}
\end{lstlisting}





\subsection{Allocators\label{sec:allocators}}
\lorem{}
