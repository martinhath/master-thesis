\chapter{Rust\label{ch:rust}}
\begin{epigraphs}
  \qitem{Rust seems sensible}{\textit{John Carmack on Twitter}}
  \qitem{If you rest, you Rust.}{\textit{Helen Hayes}}
\end{epigraphs}

Rust \cite{rust} is a new programming language focusing on safety, performance, and concurrency.
The official first stable release, Rust 1.0, was released in May 2015, and a new version of the
language as well as the official compiler, \code{rustc}, is released every 6th week. The language
is developed as an open source project on the version control platform GitHub~\cite{github} by over
2000 contributors as of May 2018~\cite{rust-github}. The Rust project is organized into
\emph{teams}, such as the Core Team, the Compiler Team, and the Documentation Team. Many of the
members of the Rust teams are Mozilla employees, and Mozilla officially sponsors the Rust project.
The language has no formal specification, although all language changes are developed and
documented through an \gls{rfc} process. For a thorough introduction to Rust, see \cite{trpl}.


\clearpage

\section{Introduction}

Rust is a compiled language with a minimal runtime, similar to C and C++. \code{rustc} uses
LLVM~\cite{llvm} as a compiler back-end for code optimization and code generation. The performance of
Rust code is very similar to that of C and C++~\cite{rust-perf}; variations are often due to the
lack of stable features like SIMD support, or from different compile time information given to LLVM
by either language.

Rust has many features from the ML family of programming languages, such as pattern matching and
tagged enums, and a rich type system with type inference. Most notably, and unlike most other
modern programming languages, Rust does not have a garbage collector. Despite this, Rust programs
does not handle memory management manually; memory management is typically done statically at
compile time by utilizing language features covered in the upcoming sections.

Rust uses \code{struct}s similar to C and C++ which can have \emph{methods}, but it does not have
inheritance. \code{Trait}s are similar to interfaces: they define methods and optionally an
implementation, and \code{struct}s \emph{implement} the \code{Trait}. Traits can even be
implemented for types that we have not defined ourselves, as long as we have defined the
\code{Trait}. This is useful, since it means we can extend types from the standard library, or from
other third party crates\footnote{A \emph{crate} is a project unit, similar to a library}.
Important \code{Trait}s include \code{Deref} (the \code{*} operator), \code{Clone} (values that are
clonable), and \code{Drop} (ran when a value is destroyed).

When an owned value leaves its scope, it is destroyed and its \code{Drop} method is ran. Primitive
types, such as \code{char} or \code{u32} does not have a \code{Drop} implementation, but types
which holds a resource, like allocated memory, often has. \code{String}  and \code{Vec<T>} are
common examples. \code{String} has a pointer to an internal buffer, which needs to be freed upon
destruction in order not to leak memory. This \code{free} call is done inside \code{String::Drop}.


\section{The Borrow Checker\label{sec:borrow-checker}}

A central concept in Rust is that of ownership. At any moment, an object has exactly one binding
which \emph{owns} the object. Ownership may be transferred (``\emph{moved}'', which is the default
behavior), or it may be \emph{lent out}. Then the receiver is \emph{borrowing} the binding.
There are two types or borrows: immutable and mutable borrows.
One of the reasons to differentiate between mutable and immutable borrows, is references in Rust
can be either aliased, or mutable, but never both.  That is, if there is a mutable reference to
some object, then that reference has to be the \emph{only} reference. This ensures that immutable
references are never changed, which makes it simpler for the programmer to reason about the code
since we get referential transparency, in addition to that it enables more compiler optimizations.

Borrowed objects are in effect \emph{references} to some data, similar to pointers or references
in other programming languages. While Rust does have raw pointers (see \cref{sec:unsafe-rust}), it
is rarely used, and passing values by reference is preferred.  The three types of ownership handling
is shown in \cref{fig:rust-ownership}. In \cref{fig:ownership-transfer} we move \code{x}, so
\code{x} is no longer usable after the last line, and an attempt to use it is caught as a compile
time error: \code{error[E0382]: use of moved value: `x`}. Since the caller of \code{foo} has
``sent'' the \code{Foo} to the function, it does no longer have to do any cleanup: this is now
\code{foo}s responsibility.

\cref{fig:immutable-borrow} shows immutable borrow of \code{x}; the function \code{foo} may use the
\code{Foo}, but it cannot mutate it. \cref{fig:mutable-borrow} shows a mutable borrow; now
\code{foo} may mutate the \code{Foo}. Note that the binding \code{x} also needs to be mutable in
order to borrow mutably.

\begin{figure}[ht]
  \begin{subfigure}{0.3\textwidth}
    \caption{Ownership transfer\label{fig:ownership-transfer}}
    \begin{lstlisting}
fn foo(f: Foo);
let x = ...
foo(x);\end{lstlisting}
  \end{subfigure}%
  \hfill
  \begin{subfigure}{0.3\textwidth}
    \caption{Immutable Borrow\label{fig:immutable-borrow}}
    \begin{lstlisting}
fn foo(f: &Foo);
let x = ...
foo(&x);\end{lstlisting}
  \end{subfigure}%
  \hfill
  \begin{subfigure}{0.3\textwidth}
    \caption{Mutable Borrow\label{fig:mutable-borrow}}
    \begin{lstlisting}
fn foo(f: &mut Foo);
let mut x = ...
foo(&mut x);\end{lstlisting}
  \end{subfigure}%
  \caption{The three types of ownership handling.\label{fig:rust-ownership}}
\end{figure}


Understanding the borrow checker is often a pain point for new programmers, and the period in which
new Rust programmers learns an intuition about how to structure programs within these rules is
often referred to as ``fighting with the borrow checker''.


\section{Lifetimes\label{sec:rust-lifetimes}}

Lifetimes is the second important concept in Rust. The idea of lifetimes is to reason about the
duration of the program execution in which some object is valid --- its lifetime. By tracking the
lifetime of all variables at compile time the Rust compiler is able to catch errors such as
returning function local variable addresses. \cref{lst:lifetimes} shows an example function
attempting to do this.  \begin{lstlisting}[style=Rust,label=lst:lifetimes]
fn foo(_a: &i32) -> &i32 {
  let num: i32 = 420;
  let r: &i32 = &num;
  r }
\end{lstlisting}
 Since Rust tracks the lifetime of all variables,
it knows that the lifetime of \code{num} is the same as that of the function body. The lifetime of
\code{r} is the same, as it is a reference to \code{num}. So when we try to return \code{r} in the
last line of the function, Rust realizes that the lifetime of the reference we return ends its life
at the end of the function; this is clearly not what we wanted, since it would make the returned
reference dead on arrival. Compilation fails with the following error: \code{error[E0597]: `num`
does not live long enough}.


Although Rust programmers may have to think about the lifetime of the variables, they seldom have
to write lifetime annotated functions, due to \emph{lifetime elision} --- the compiler can usually
figure out the most general lifetime that fits the function. Functions may be annotated with
explicit lifetimes, for instance if it takes multiple references in which the relative difference
of the lifetimes of the references is important.
\code{struct}s can also be annotated with lifetimes, and in fact is required to be so if any of its
members are references. This is because the lifetime of the struct is bounded by the lifetime of
its member variables.
\begin{lstlisting}[style=Rust]
struct Person<'a> {
  age: i32,
  name: &'a str }
\end{lstlisting}
Should we have a function that crates a new \code{Person} we might want to annotate it explicitly,
if the function takes multiple references, but only one of these references is the \code{name}
field:
\begin{lstlisting}[style=Rust]
fn make<'x, 'y>(f: &'y File, n: &'x str) -> Person<'x> { ... }
\end{lstlisting}
This way we can convey the information that the resulting \code{Peron} should live as long as
\code{n}, but may outlive the file \code{f}.



\section{Unsafe Rust\label{sec:unsafe-rust}}

When talking about the Rust programming language, one usually talks about a subset of Rust, called
\emph{Safe Rust}. In Safe Rust, there are no race conditions, mutable memory locations are never
aliased, and all pointer accesses are valid.  The real world, on the other hand, rarely offers
these guarantees, and the unfortunate truth which Rust programmers must deal with is that in order
to implement some of these safe abstractions we want (like \code{Vec}, \code{Mutex}, and
\code{Box}), some unsafety is required.  For this reason, Rust offers an escape hatch for some of
its rules: \emph{Unsafe Rust}.

The difference between Safe and Unsafe Rust is only four things. In Unsafe Rust one may:
\begin{enumerate*}[1) ]
    \item dereference raw pointers
    \item mutate statics
    \item call \code{unsafe} functions
    \item implement \code{unsafe} traits.
\end{enumerate*}
One way of thinking about the unsafety of ones codebase is that there should be no undefined
behavior in safe code, no matter how the code looks like. In other words, it should be impossible
to mess up so badly as to invoke undefined behavior without typing \code{unsafe}.

Dereferencing raw pointers is naturally \code{unsafe}, as it is not possible to statically
guarantee that the address of the pointer is valid memory, or that the objects it points to is
still alive, nor that mutation of that memory does not change an immutable reference some other
place in the program. Mutation of \code{static} variables is also unsafe due to mutability of
aliased references, and due to the lack of thread synchronization.

\code{unsafe} functions and traits are just a marker added to the function or trait, signaling that
not all uses of this is guaranteed to be safe. As an example, the trait \code{Send} is a marker
trait and types implementing \code{Send} may be sent across thread boundaries. While this is fine
for most types, there are types which does not allow this. The reference counted pointer
\code{Rc<T>} is an example, which is a pointer to a tuple\footnote{Not really, but for our purposes
here we can pretend that it is.} \code{(count, data)}. The \code{count} is incremented each time
\code{.clone()} is called, and decremented when a variable is \code{Drop}ped.  To understand why
this cannot be send across thread boundaries safely, consider what happens if $T\sb{1}$
\code{.clone()} at the same time as $T\sb{2}$ \code{Drop}s it: the \code{count} field is written to
twice without any synchronization atomic operations\footnote{\code{Rc} does not use atomics for
performance reasons, but \code{Arc} does, and it does implement \code{Send}.} --- a race condition!

Rust is marketed as a safe programming language; it is however important to realize that this is
only a half-truth. In principle Rust, due to the \code{unsafe} keyword, is no more safe than any
codebase in C or C++ is, and third party libraries might hide the fact that they utilize
\code{unsafe} code in order to appear more ``safe''. The language offers many ways to avoid having
to type the dreaded six letters and enter the world where all bets are off, but nobody is stopping
crate authors, co-workers, or even yourself, to write \code{unsafe} code.



\section{Concurrency}

One of the main focuses of Rust is concurrency, and the language does offer a helping hand in
writing concurrent code; many of these arises naturally from the type system, and the ownership
model, like the single owner principle, and the single mutable reference rule.


\subsection{Concurrency and Aliasing\label{sec:concurrency-and-aliasing}}

One observation to make from the reference rules as presented in \cref{sec:borrow-checker} is that
since references are either aliased or mutable, then there can be no writes shared data between
threads, in Safe Rust, even using atomics. While this is \emph{technically} true, the Rust standard
library uses \code{\&T} and \code{\&mut T} slightly different than ``immutable'' vs ``mutable'' in
this context: \code{\&T} means that the type may be shared between threads.

Take \code{AtomicUsize} as an example, a \code{usize} exposing atomic operations like \code{store},
\code{load}, and \code{compare_and_swap}, which signatures are shown in \cref{lst:atomicusize}.
\begin{lstlisting}[label=lst:atomicusize,caption=Signatures for selected operations on
\code{AtomicUsize}]
fn load(&self, order: Ordering) -> usize;
fn store(&self, val: usize, order: Ordering);
fn swap(&self, val: usize, order: Ordering) -> usize;
fn compare_and_swap(&self, current: usize, new: usize, order: Ordering) -> usize;
\end{lstlisting}

Clearly, \code{AtomicUsize::store} modifies memory of the \code{usize}; despite this the function
is \code{\&self} and not \code{\&mut self}, since the operation is allowed on variables which are
shared between threads.
This is a useful distinction, since we can have methods on \code{AtomicUsize} that \emph{is}
\code{\&mut self}, which then is only possible to invoke should the variable not have been shared
between threads yet; we know this since this means that we have aliased mutable references, which
is not allowed. For instance, \code{AtomicUsize::get_mut(\&mut self) -> \&mut usize} allows the
underlying \code{usize} to be changed without any synchronization overhead.


\subsection{Common Patterns}

The standard library's synchronization module \code{std::sync} contains primitives that most
concurrent programs require, such as \code{Mutex}, \code{Channels}, \code{Condvar}, and
\code{Atomic}s. A common pattern in Rust is the \gls{raii} pattern. The idea is that resources
should be managed automatically when constructing and destructing an object. \code{Mutex} uses
these ideas: \code{Mutex::lock} returns an \code{Result<MutexGuard>}, where the \code{MutexGuard}
wraps a mutable reference to the data that is protected by the \code{Mutex}. When the
\code{MutexGuard} goes out of scope, its \code{Drop} implementation is ran, and the \code{Mutex} is
unlocked.

It is common among Rust programmers to build abstractions over lower level primitives. For
instance, a common pattern in parallel and concurrent programming is to have a \emph{thread pool},
which is given work, and internally handles the thread synchronization and work division. Example
usage of such an abstraction could be \code{let tp = ThreadPool::new(); tp.execute(|| { ... });}.
Since this can be implemented without any special compiler support, such crates are usually made as
third party libraries.

Another example is data parallelism: given some collection of data we want to iterate over the
elements and perform some operation on each element. The Rust library \code{rayon} offers exactly
this: parallel iterators. Instead of writing \code{vec.iter()} to iterate over a \code{Vec} and
then performing some operation on each element sequentially, with \code{rayon} we can write
\code{vec.par_iter()}, and get data parallelism for free. The operation is then ran in parallel
with any number of threads. Internally \code{rayon} uses a thread pool and work stealing to handle
the division of labor among the threads.


\section{Nightly Rust}

The Rust language and compiler follows a fixed release schedule, where a new stable version is
released every six weeks. In addition to this there is the beta branch, which is the upcoming
version, and the nightly version which is the most recent version, build daily from the
\code{master} branch of the source tree.

The nightly version of the compiler allows users to opt in on \emph{unstable} features: features
that are partially or fully implemented, but which details are not yet committed to. These features
includes new APIs in the standard library, new syntax, and new language features all together.
As we have used multiple unstable features in CMR, we look at some of them in detail.


\subsection{Non-Lexical Lifetimes\label{sec:nll}}
The current implementation of lifetime checking in the compiler is \emph{lexical}, meaning
variables are live until they go out of scope, despite not being used. This is a limitation that
one may want to get rid off. The feature \gls{nll} lifts this requirement, and lets the lifetime of
a variable last only until its last usage. Having this it is possible to seemingly break some of
Rust rules, like aliased mutable references:
\begin{lstlisting}[style=Rust]
let mut v = vec![1,2,3];
let r1 = &mut v;
let r2 = &mut v;
\end{lstlisting}

This clearly violates one of the Rust rules, namely that we cannot have mutable aliased references.
Yet, in this example we have two mutable references, \code{r1} and \code{r2}, to the same data.
With \gls{nll} this will compile, as we do not use \code{r1} after having made \code{r2}, so its
lifetime is implicitly ended right after its declaration. If we write \code{r1.push(1);} after
\code{let r2}, we get the same error as without using \gls{nll}, since the lifetime \code{r1}
overlaps with the lifetime of \code{r2}.



\subsection{Trait Objects\label{sec:trait-objects}}

When using traits in function signatures or structs we can either make the struct generic over some
type that implements the trait, or we can use dynamic dispatch. As generics usually are implemented
by copying the source code for the type for each invocation of a new type, it increases code size
and compilation time. In addition, collections and similar structures cannot mix different types: a
\code{Vec<SomeTrait>} cannot both contain elements of type \code{A} and \code{B}, even if both
implements \code{SomeTrait}.

Dynamic dispatch is the other option. Now variables are \emph{fat pointers}, containing both the
pointer to the data type, and a pointer to a \code{vtable}\footnote{the name \code{vtable} comes
from the C++ world, where function on abstract types are called \code{virtual} functions}, which
contains information about the function addresses for that type, as shown in
\cref{fig:trait-objects}. The entry in the \code{vtable} is all functions for some trait. With this
we can take any concrete type, and follow its vtable pointer, in order to find the implementation
of some trait function for that type. In \cref{fig:trait-objects}, both \code{Foo} and \code{Bar}
implements some trait which have a function named \code{fnc}. By following the pointers from the
stack, we get the data (left) and the function pointer (right). This way of implementing Trait
Objects are usually not mandated by any standard, but it is popular across different language
implementations nevertheless.

\begin{figure}[ht]
  \centering
  \begin{tikzpicture}[node distance=1cm]
\tikzset{lnode/.style={node distance=1.25cm, rectangle split, rectangle
         split horizontal, rectangle split parts=2, draw,font=\footnotesize}}

  \node [lnode]  (st1)                {bing \nodepart{second} bong};
  \node [lnode]  (st2) [below of=st1] {bing \nodepart{second} bong};
  \node [draw, inner sep=0mm, fit={(st1) (st2)}] {};
\end{tikzpicture}

  \caption{Illustration of memory when using Trait Objects.\label{fig:trait-objects}}
\end{figure}

While trait objects offers greater flexibility in the usage of traits, the pointer jumping leads to
worse cache behavior which may have a large impact on performance, and important compiler
optimizations like in-lining is impossible.



\subsection{Specialization\label{sec:specialization}}

Specialization is a feature which allows multiple implementations of a trait for the same type,
where the implementations are ordered by their specificity.

Assume we want to implement the trait \code{Debug} for a struct that is generic over some type
\code{T}: \code{Struct<T>}. We might want to have different implementations of \code{Debug}
depending on whether the generic parameter \code{T} implements \code{Debug} or not.
Specialization makes this possible.
\begin{lstlisting}[
  style=Rust,
  label=lst:specialization,
  caption=Using specialization to implement a trait twice.,]
impl<T> Trait for Struct<T> {
  default fn fmt(&self);
}
impl<T: Debug> Trait for Struct<T> {
  fn fmt(&self);
}
\end{lstlisting}


With only these two implementation it is clear which of the two we want for any type: if \code<T>
implements \code{Debug} we want the second, and if it does not, we want the first. However, if we
mix in yet another trait, \code{Clone}, such that we have a third implementation
\begin{lstlisting}[style=Rust]
impl<T: Clone> Trait for Struct<T> { ... }
\end{lstlisting}
it is no longer clear which implementation to use if \code{T} implements \emph{both} \code{Clone}
and \code{Debug}. The current implementation forbids such specializations.


\subsection{Allocators\label{sec:allocators}}

The final nightly feature that we look at is \emph{allocators}. It is not yet possible to change
the default allocator in stable Rust, but a suggested API for creating new allocators and
specifying the default system wide allocator for Rust programs is available by opting in on the
allocator feature. The feature defines a trait \code{GlobalAlloc} that defines functions analogous
to \code{malloc} and \code{free} from libc, and a attribute \code{\#[global_allocator]} to select
which allocator we want to use.

The default allocator for Rust is jemalloc~\cite{jemalloc}. By using other external crates we can
use either the default system allocator, or jemalloc wrapped in our own allocator. This can be
useful if we want to do bookkeeping, gather statistics, or do any thread synchronization outside of
the actual allocator we are using.

\begin{figure}[ht]
\begin{lstlisting}[caption=Custom allocators wrapping jemalloc and the system allocator]
pub struct WrapJemalloc;
unsafe impl GlobalAlloc for WrapJemalloc {
    unsafe fn alloc(&self, layout: Layout) -> *mut Opaque {
        (*\lit{Do something before calling alloc}*)
        Jemalloc.alloc(layout) }
    unsafe fn dealloc(&self, ptr: *mut Opaque, layout: Layout) {
        (*\lit{Do something before calling free}*)
        Jemalloc.dealloc(ptr, layouer); } }

pub struct WrapSystem;
unsafe impl GlobalAlloc for WrapSystem {
    unsafe fn alloc(&self, layout: Layout) -> *mut Opaque {
        (*\lit{Do something before calling alloc}*)
        System.alloc(layout) }
    unsafe fn dealloc(&self, ptr: *mut Opaque, layout: Layout) {
        (*\lit{Do something before calling free}*)
        System.dealloc(ptr, layout); } }
\end{lstlisting}

\end{figure}
