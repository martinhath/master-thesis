\chapter{Rust\label{ch:rust}}


\section{Introduction}
\blindtext{}



\section{The Borrow Checker\label{sec:borrow-checker}}

A central concept in Rust is that of ownership. At any moment, an object has exactly one binding
which \emph{owns} the object. Ownership may be transfered (``\emph{moved}'', which is the default
behaviour), or it may be borrowed, of which there are two types: immutable borrows and mutable
borrows. The borrowed binding is in effect a \emph{reference} to the data, similiar to references
in other programming languages. The three types of ownership handling is shown in
\cref{fig:rust-ownership}.

\begin{figure}[ht]
  \begin{subfigure}{0.3\textwidth}
    \caption{Ownership transfer\label{fig:ownership-transfer}}
    \begin{lstlisting}
fn foo(f: Foo);
let x = ...
foo(x);\end{lstlisting}
  \end{subfigure}%
  \hfill
  \begin{subfigure}{0.3\textwidth}
    \caption{Immutable Borrow\label{fig:immutable-borrow}}
    \begin{lstlisting}
fn foo(f: &Foo);
let x = ...
foo(&x);\end{lstlisting}
  \end{subfigure}%
  \hfill
  \begin{subfigure}{0.3\textwidth}
    \caption{Mutable Borrow\label{fig:mutable-borrow}}
    \begin{lstlisting}
fn foo(f: &mut Foo);
let mut x = ...
foo(&mut x);\end{lstlisting}
  \end{subfigure}%
  \caption{The three types of ownership handling.\label{fig:rust-ownership}}
\end{figure}


One of the reasons to differentiate between mutable and immutable borrows, is references in Rust
can be either aliased, or mutable. That is, if there is a mutable reference to some object, then
that reference has to be the \emph{only} reference. This ensures that immutable references are
never changed.



\section{Lifetimes\label{sec:rust-lifetimes}}
\blindtext{}



\section{Unsafe Rust}

When talking about the Rust progrmaming language, one usually talks about a subset of Rust, called
\emph{Safe Rust}. In Safe Rust, there are no race conditions, mutable memory locations are never
aliased, and all pointer accesses are valid.  The real world, on the other hand, offers seldom
these guarantees, and the unfortunate truth which Rust programmers must deal with is that in order
to implement some of these safe abstractions we want (like \code{Vec}, \code{Mutex}, and
\code{Box}), some unsafety is required.  For this reason, Rust offers an escape hatch for some of
its rules: \emph{Unsafe Rust}.

The difference between Safe and Unsafe Rust is only four things. In Unsafe Rust one may:
\begin{enumerate*}[1) ]
    \item dereference raw pointers
    \item mutate statics.
    \item call \code{unsafe} functions
    \item implement \code{unsafe} traits
\end{enumerate*}

Dereferencing raw pointers is naturally \code{unsafe}, as it is not possible to statically
guarantee that the address of the pointer is valid memory, or that the objects it points to is
still alive. Mutation of \code{static}  variables is also unsafe due to the lack of thread
synchronization.

\code{unsafe} functions and traits are just a marker added to the function or trait, signaling that
not all uses of this is guaranteed to be safe. For instance, the trait \code{Send} is a marker
trait, and types implementing \code{Send} may be sent across thread boundaries. While this is fine
for most types, there are types which does not allow this. The reference counted pointer
\code{Rc<T>} is an example, which is a pointer to a tuple \code{(count, data)}. The \code{count} is
incremented each time \code{.clone()} is called, and decremented when a variable is \code{Drop}ped.
To understand why this cannot be send across thread boundaries safely, consider what happens if
$T\sb{1}$ \code{.clone()} at the same time as $T\sb{2}$ \code{Drop}s it: the \code{count} field is
written to twice without any synchronizationor atomic operations\footnote{\code{Rc} does not use
atomics for performance reasons, but \code{Arc} does, and it does implement \code{Send}.} --- a race
condition!

One way of thinking about the unsafety of ones codebase is that there should be no undefined
behaviour in safe code, no matter how the code looks like. In other words, it should be impossible
to mess up so badly as to invoke undefiend behaviour without typing \code{unsafe}.



\section{Concurrency}

One observation to make from the reference rules as presented in \cref{sec:borrow-checker} is that
since references are either aliased or mutable, then there can be no writes shared data between
threads, in Safe Rust, even using atomics. While this is \emph{technically} true, the Rust standard
library uses \code{\&T} and \code{\&mut T} slightly different than ``immutable'' vs ``mutable'' in
this context: \code{\&T} means that the type may be shared between threads.

Take \code{AtomicUsize} as an example, a \code{usize} exposing atomic operations like \code{store},
\code{load}, and \code{compare_and_swap}, which signatures are shown in \cref{lst:atomicusize}.
\begin{lstlisting}[label=lst:atomicusize,caption=Signatures for selected operations on
\code{AtomicUsize}]
fn load(&self, order: Ordering) -> usize;
fn store(&self, val: usize, order: Ordering);
fn swap(&self, val: usize, order: Ordering) -> usize;
fn compare_and_swap(&self, current: usize, new: usize, order: Ordering) -> usize;
\end{lstlisting}

Clearly, \code{AtomicUsize::store} modifies memory of the \code{usize}; despite this the function
is \code{\&self} and not \code{\&mut self}, since the operation is allowed on variables which are
shared between threads.



\section{Nightly Rust}
\blindtext{}


\subsection{Non-Lexical Lifetimes\label{sec:nll}}
\blindtext{}


\subsection{Trait Objects\label{sec:trait-objects}}
\blindtext{}


\subsection{Specialization\label{sec:specialization}}
\blindtext{}


\subsection{Allocators\label{sec:allocators}}
\blindtext{}
