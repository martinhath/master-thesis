\chapter{Usage of CMR}
\blindtext{}

\section{Stack}

\begin{figure}[b]
\centering
\begin{tikzpicture}
\tikzset{lnode/.style={node distance=2cm, rectangle split, rectangle
         split horizontal, rectangle split parts=2, draw, rounded
         corners=0.05cm,font=\footnotesize,fill=white}}

  \node [draw] (head)  {\code{head}};
  \node [lnode, node distance=3cm] (n1) [right of=head] {1};
  \node [lnode] (n2) [right of=n1] {4};
  \node [lnode] (n3) [right of=n2] {1};

  \draw [circlebase,-latex] ($(n1.center)!0.5!(n1.east)$) -- (n2);
  \draw [circlebase,-latex] ($(n2.center)!0.5!(n2.east)$) -- (n3);
  \draw [circlebase,-latex] ($(n3.center)!0.5!(n3.east)$) -- ($(n3) + (1,0)$);

  \node [lnode] (t1n) at ($(head.east)!0.5!(n1.west) + (0, 1.5)$) {3};
  \draw[circlebase,-latex] ($(t1n.center)!0.5!(t1n.east)$) -- (n1);
  \node (t1) [left of=t1n] {$T\sb{1}$};

  \node [lnode] (t2n) at ($(head.east)!0.5!(n1.west) - (0, 1.5)$) {4};
  \draw[circlebase,-latex] ($(t2n.center)!0.5!(t2n.east)$) -- (n1) ;
  \node (t2) [left of=t2n] {$T\sb{2}$};


  \draw[-latex,color=theme1!15] (head.east) to [out=0,in=198] ($(t1n.south)!0.8!(n1.west)$);
  \draw[-latex,color=theme1!30] (head.east) to [out=0,in=216] ($(t1n.south)!0.6!(n1.west)$);
  \draw[-latex,color=theme1!45] (head.east) to [out=0,in=234] ($(t1n.south)!0.4!(n1.west)$);
  \draw[-latex,color=theme1!60] (head.east) to [out=0,in=252] ($(t1n.south)!0.2!(n1.west)$);
  \draw[-latex,color=theme1!75] (head.east) to [out=0,in=270] (t1n.south);

  \draw[-latex,color=theme2!15] (head.east) to [out=0,in=161] ($(t2n.north)!0.8!(n1.west)$);
  \draw[-latex,color=theme2!30] (head.east) to [out=0,in=144] ($(t2n.north)!0.6!(n1.west)$);
  \draw[-latex,color=theme2!45] (head.east) to [out=0,in=126] ($(t2n.north)!0.4!(n1.west)$);
  \draw[-latex,color=theme2!60] (head.east) to [out=0,in=108] ($(t2n.north)!0.2!(n1.west)$);
  \draw[-latex,color=theme2!75] (head.east) to [out=0,in=90] (t2n.north);

  \draw [-latex] (head) -- (n1);
\end{tikzpicture}

\caption{$T\sb{1}$ and $T\sb{2}$ both tries to swap the \code{head} pointer towards their
node.}
\end{figure}

We begin by looking at an implementation of a conurrent stack, which is arguably the simplest
concurrent data strucutre. The definitions of the \code{Stack} and \code{Node} structs and the two
most important operations on a stack, \code{push} and \code{pop}, is shown in \cref{lst:stack-ops}.
We look at each one in turn. Construction of the stack is omitted for brevity.

\begin{figure}[ht]
\begin{lstlisting}[caption=\code{Stack::push} and \code{Stack::pop},label=lst:stack-ops,
    numbers=left, numberstyle=\color{gray}\ttfamily{}ST,
]
struct Stack<T> { top: SharedGuard<Node<T>>, }
struct Node<T> { data: ManuallyDrop<T>, next: Atomic<Node<T>>, }

impl<T> Stack<T> {
    pub fn push(&self, t: T) {
        guards!(_new_top, _top);
        let mut new_top = cmr::alloc(_new_top, Node::new(t));
        loop { let top = cmr::guard(_top, &self.top);
               unsafe { new_top.get_mut().next = Atomic::new(top); }
               if self.top.cas(top, new_top, SeqCst).is_ok() { break; } } }

    pub fn pop(&self) -> Option<T> {
        guards!(_top, _next);
        loop { let top = cmr::guard(_top, &self.top).ptr()?;
               let next = cmr::guard(_next, &top.next);
               if self.top.cas(top, next, SeqCst).is_ok() {
                   let node = unsafe { top.move_out() };
                   return Some(node.data()); } } } }
\end{lstlisting}

\end{figure}

\subsubsection{Push}

\fixme{After writing \ref{sec:common-patterns}, revisit this}
\code{push} allocates the stack node itself, so it takes the value we want to push onto the stack
\coderef{ST5}. We start out by declaring two \mc{Guard}s \coderef{ST6}: one for the new node we
allocate, and one for the head of the stack. We must protect the head of the stack, since the node
may be removed after we read its address, and we would have a dangling pointer.  Next we allocate a
new \code{node} \coderef{ST7}, which is done outside the retry loop so that we only have to
allocate one time per call to \code{push}. Now we enter the retry loop, which we repeat until we
succeed in changing the top pointer of the stack to our new node.
The top node is read \coderef{ST8}, and the \code{next} pointer of the new node is set to the head
\coderef{ST9}. If we succeed of chaning the \code{top}  pointer of the stack to our new node, we
break out of the loop and return \coderef{ST10}. If not, we retry until we do.


\subsubsection{Pop}

\code{pop} is similar to \code{push}. We declare two \mc{Guard}s \coderef{ST13}, but this time they
are for the first and second node in the stack. We read the \code{top} pointer \coderef{ST14}, and
return from the function if it is \code{null} using the \code{?} Rust operator. We then read the
next pointer of the node \coderef{ST15}; here the \code{null} case is the same as the
non-\code{null} case. We try to swap the head pointer from the first to the second node
\coderef{ST16}; if we fail we restart, and if we succeed we move out the \code{Node} from the
\mc{Guard}. This is an \code{unsafe} operation, as the type is copied out of its original place,
effectively aliasing it. At last, the data is returned.

As an example of why reading and returning the node data is \code{unsafe} in the general case,
consider two threads $T\sb{1}$ and $T\sb{2}$ using a \code{Stack<Box<T>>}. $T\sb{1}$ is looking at
a node $n$, and $T\sb{2}$ is \code{pop}ping $n$ from the stack, getting the \code{Box<T>} back from
it. Now $T\sb{2}$ \code{drop}s the \code{Box}, which frees the pointer. If $T\sb{1}$ decides to
look at the data in $n$, it will dereference a freed pointer, which is a use-after-free
(\cref{def:use-after-free}). \fixme{fix this argument please} Despite being \code{unsafe} in the
general case, it is safe for the implementation of the stack as presented, since no operation on
the stack looks at the data of a node, except in \coderef{ST17}, where only one thread may be for
any given node, since we succeed the \code{cas} operation.


\section{Michael-Scott Queue}
\fixme{refs here pls}
skra

\begin{figure}[b]
\centering
\begin{tikzpicture}
\tikzset{lnode/.style={node distance=2cm, rectangle split, rectangle
         split horizontal, rectangle split parts=2, draw, rounded
         corners=0.05cm,font=\footnotesize}}
  \tikzset{struct/.style={rectangle split, rectangle split horizontal, rectangle split parts=2,
                          draw,
  }}

  \node [struct] (struct) {\code{head}\nodepart{second}\code{tail}};

  \node [lnode, node distance=1.2cm] (n2) [below of=struct] {3};
  \node [lnode,rectangle split part fill={gray, white}] (n1) [left of=n2]      {\null};
  \node [lnode] (n3) [right of=n2]     {1};
  \node [lnode] (n4) [right of=n3]     {4\nodepart{second}$\bot$};

  \draw [circlebase,-latex] ($(n1.center)!0.5!(n1.east)$) -- (n2);
  \draw [circlebase,-latex] ($(n2.center)!0.5!(n2.east)$) -- (n3);
  \draw [circlebase,-latex] ($(n3.center)!0.5!(n3.east)$) -- (n4);
  % \draw [circlebase,-latex] ($(n4.center)!0.5!(n4.east)$) -- ($(n4) + (1,0)$);

  \draw [-latex] (struct.west) -| (n1.north);
  \draw [-latex] (struct.east) -| (n4.north);

\end{tikzpicture}

\caption{The Michael-Scott Queue. The first node in the queue is a sentinel node.}
\end{figure}

\section{Split-Order List}
\blindtext{}

\section{Concurrent HashMap}
\blindtext{}

\section{Crossbeam Integration}
\blindtext{}
