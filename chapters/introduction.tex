\chapter{Introduction}

In this chapter we introduce the problem space in which this thesis operates.
We look at the general problem that we want to solve, and why it is an interesting problem.
We also draw an outline of the structure of this thesis, and summarizes each chapter in short.

\clearpage



\section{History}


The clock speed of computer processors have had a very fast increase in the last 50 years;
Gordon Moore observed in 1965 that he number of components in an integrated circuit roughly had
doubled every 18 months, and speculated that this trend would continue for the next 10 years.
The law, which now is knows as Moores Law, still holds true today.
Moores Law has also been used on the clock speed of CPUs. This trend, however, has grinded to a
halt; the clock speed of desktop processors in the last 10 years have been more or less stagnant.
Despite the lack of clock increase, there have been major improvements in modern CPUs. Improvements
include pipelining, branch prediction, out-of-order execution, and, most important for this thesis,
multiple processing cores.


Multi-core processors have been increasingly mainstream in the last 10 years; modern enthusiast
desktop CPUs, like the recently announced AMD ThreadRipper 2, has 32 cores and support for 64
hardware threads. Even embedded systems, like smartphones, often come in variants with 4 or 8
threads. While multi-core systems may offer increase on computation speed, they also introduce new
problems; utilization of parallel systems is not trivial. Many problems are inherently serial and
serial solution to problems are often much easier to develop than efficient parallel systems.
Efficient synchronization between processes is also, perhaps surprisingly, a difficult problem.


Many modern programming languages aim for developer productivity. Many programming abstractions and
runtime subsystems are introduced in the name of convenience; one of these is a memory management
system, often refered to as a garbage collector. Despite the developer ergonimic inprovements such
systems claim to improve, they often come at a cost: efficiency. Garbage collectors have to support
a wide range of use cases, like both short-lived allocations of small objects \emph{and} bulk
allocation of large objects, the overhead of a memory manager may be far from optimal.



\subsubsection{Rust}

Rust is a new programming language aiming to unify the developer ergonomics promised by managed
languages, and the efficiency of unmamaged languages. The work of managing memory is left to the
compiler, which at compile time analyses the program, and enforces certain rules about how memory
is handled. This way the programmer does not need to manage memory in the traditional sense,
although it does impose a cognitive overhead when developing programs, which larger than that of managed
languages. One of the features of Rust that may make it viable for systems programmers with strong
requirements for performance and stability is that the memory management system in Rust is fully
controlled by the programmer, as there is no runtime.


\section{This Thesis}

In this thesis we aim to develop a concurrent memory management system for Rust called CMR\@, which
is based on Forkscan~\ref{sec:forkscan}. We believe this is an interesting topic since it is not
clear how to incoorporate memory management for concurrent system in the ownership model that Rust
provides, although there are existing projects which does similiar things; these projects are often
implementation of other known memory reclamation schemes like Hazard Pointers and Epoch-Based
reclamation. This thesis develops a new scheme.

Despite one of the promises that Rust makes is one of predictability and control, CMR works more
like a traditional garbage collector. However, as the scheduling of threads in a process inherently
is fuzzy and unpredictable, we believe that this is an intrinsic property of any concurrent memory
reclamtion system.

The aims for this thesis is to consider CMR as an alternative to more traditional methods for
memory reclamation. The viability of such a management system is both performance and usability,
and we will comment both.


\section{Outline}

We mention briefly the contents of each chapter. The chapters are also prefaced with a more fine
grained introduction.

\cref{ch:background} highlights the most important background material that is needed in order to
appreciate the contents of this thesis. Much of the material is a part of a standard computer
science curriculum, but we repeat it here nevertheless. Relevant sections from this chapter is
cited in the remaining of the text.

\cref{ch:rust} introduces the reader to the Rust programming language, in which the impementation
of CMR is written. The way Rust functions makes it an integral part of how CMR is designed. While
this could be considered background, itis its own chapter due to its importance for this project as
a whole.

\cref{ch:cmr} presents the memory management system CMR at a high level.  We look at primitives of
CMR and common operations such as allocation and reclamation protection.  The chapter presents CMR
on a high level in order to clearly differentiate the challenges of CMR from the challenges of
implementing CMR on a real system in a real programming language. There is no real code in this
chapter.

\cref{ch:implementation} describes the implementation of CMR in Rust, including the most important
primitives and procedures. With the exception of a few omitted code sections, which are clearly
marked as such, all program code in this chapter is fully functionally Rust code taken from the
impementation.

\cref{ch:usage} contains example usage of CMR for external applications. We have implemented four
data strucutres using CMR, and these four along with their implemetations are discussed further in
this chapter.

\cref{ch:methodology} mentiones practical matters when it comes to testing and benchmarking of the
system. This is included in the thesis as it may be quite different from testing and benchmarking
of sequential systems, or systems written at a higher abstraction layer.

\cref{ch:results} discusses experimental results of both operations from CMR as well as the data
structures from the previous chapter. We look at the performance on a range of different machines,
from affordable desktop computers to high-end multiple socket server systems.

\cref{ch:conclusion} concludes this thesis with a short summary of the results and discussion about
the applicability and usability of the system as presented, in addition to suggestions for future
work.
