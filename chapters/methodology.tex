\chapter{Methodology\label{ch:methodology}}

This chapter summarizes some of the practical matters surrounding the implementation part of the
thesis. This is neither a description of, nor a part of the implementation, of CMR\@. Despite this
we believe documenthing the methods of experimentation is of equal importance in Computer Science
as in any other science.

We look at testing in \cref{sec:methods-testing}, as testing of concurrent systems are of a
different nature than that of sequential programs.  \cref{sec:methods-benchmarking} shows how we
have performed the benchmarks, which results are presented in \cref{ch:results}.


\clearpage

\section{Testing\label{sec:methods-testing}}

Testing is an important part of software development. While formal methods have not yet made its
way into the software development insdustry, simpler and more heuristic methods, like unit testing
and integration testing, are widespread. However, while testing is useful to improve the quality of
software, it is far from sufficient. As Edsger Dijkstra famously said~\cite{buxton1970software}:
\begin{displayquote}
Testing shows the presence, not the absence of bugs
\end{displayquote}

In the world of concurrent programming this is even less so. Many bugs are manifested through
unfortunate\footnote{Some would call interleavings that reveal bugs fortunate} thread execution
interleavings done by the scheduler. We try to reveal these interleavings by repeatedly running
tests until our confidence that no such interleavings exists is sufficiently high. In addition, we
run tests with tools such as Valgrind~\cite{valgrind} and our own sanitizer (\cref{sec:sanitizer}).
Tests were also ran with and without compiler optimizations, as these optimizations often reveal
yet more bugs.


\subsection{Sanitizer\label{sec:sanitizer}}

To automate validation of pointer reads we made a compile time feature\footnote{\emph{features} are
similiar to \code{\#ifdef}s in C and C++} called \code{sanitize} that tracked all allocations,
frees, and pointer reads.  Allocations and frees were tracked in two \code{HashMap}s,
\code{ALLOCATIONS} and \code{FREES}.  On each new allocation, we insert it into the \code{HashMap}
while asserting that it was not there previously. We also remove it from the frees map, in case it
had previously been allocated and freed. Since we are using a custom pointer type, \code{Ptr},
checking the validity on each pointer access is possible, as shown in the snippets below:

\begin{lstlisting}[style=Rust]
pub fn alloc<'a, T: Trace>(t: T) -> Ptr<'a, T> {
    let addr = B::into_raw(B::new(t)) as usize;
    #[cfg(feature = "sanitize")] {
        let mut a = ALLOCATIONS.lock().unwrap();
        assert!(a.insert(addr));
        let mut f = FREES.lock().unwrap();
        f.remove(&addr); }
    unsafe { Ptr::new(addr) } } \end{lstlisting}

\begin{lstlisting}[style=Rust]
impl<'a, T> Deref for Ptr<'a, T> {
    type Target = T;
    fn deref(&self) -> &T {
        #[cfg(feature = "sanitize")] {
            let a = ::alloc::ALLOCATIONS.lock().unwrap();
            if !a.contains(&addr(self)) {
                let was_freed = ::alloc::FREES.lock().unwrap().contains(&addr(self));
                panic!("{:x} is not valid. Was it freed? {}", self.data, was_freed); } }
        unsafe { &*(self.as_raw()) } } }
\end{lstlisting}


This feature was of great help during development and testing, and it quickly reported illegal
memory accesses, without having the overhead of other sanitizers like Valgrinds \code{memcheck}.
The downside of this sanitizer is that it did not show the source of the errors, only their
precense. However, with this in mind we could rerun the program using \code{memcheck} and try to
force the illegal accesses to manifest themselves, as we then knew that they were present.



\section{Benchmarking\label{sec:methods-benchmarking}}

The benchmarks are ran with timed trials, where a function is ran repeatedly for a specified
duration with any number of threads. The number of executions is counted for each thread and the
total operations per second is summed and reported. All threads run the same code, but they may
have different thread local data. This is useful when benchmarking \code{HashMap::insert}, so that
the threads can inserts values with different keys. Threads also time their execution time to catch
skewage in the executed wall time for each thread, due to unfortunate scheduling.

There are a number of pitfalls when it comes to benchmarking code. We discuss a few of them;
\cite{rare} is a good resource for experimental testing of data structures.

Initialization of data structures should not be done on a single core as this creates
a strong skew of data locality for that core, and other cores will have reduced performance due to
the data locality. This is especially important on NUMA systems with multiple CPU sockets.
The effect of having a single thread initializing all data is also dependent on the allocator used.

Having a constant overhead, and assuming that all workloads are equally hit by the overhead of the
performance profiling system may also lead to errors; smaller workloads will naturally be more
affected, and percentage wise changes to the reported data may get biased.


\subsection{Trench}


In order to more effectively benchmark threaded applications in Rust, an open source benchmarking library
called \code{trench}~\cite{trench} was developed. The library handles thread management and state
for the runs of the benchmark. Trench supports both mutable thread local state and immutable shared
state between all threads.

\begin{figure}[ht]
\begin{lstlisting}[style=Rust,
label=lst:trench-ex,caption=\code{Hashmap::insert} benchmark using \code{trench}]
fn hashmap_insert(num_threads: usize) {
    fn func(state: &HmState, local: &mut RandomSource<u64>) {
        state.hashmap.insert(local.next(), 0); }
    let b = trench::TimedBench::<HmState, RandomSource<u64>>::with_threads(num_threads);
    b.with_local_state(|l| { cmr::thread_activate();
                             l.gen_n(10_000_000); });
    let res = b.run_for(duration(), func);
    b.with_local_state(|_| cmr::thread_deactivate());
    println!("cmr::HashMap\tinsert\t{} ops/sec", fmt_thousands_sep(res.ops_per_sec)); }
\end{lstlisting}
\end{figure}


\begin{figure}[ht]
  \begin{tikzpicture}[node distance=1cm]
  \tikzset{n/.style={draw, minimum height=4mm, minimum width=4mm, rounded corners=0.5mm}}
  \tikzset{sentinel/.style={fill=gray}}
  \tikzset{arr/.style={draw, minimum height=5mm, minimum width=5mm, inner sep=0cm, outer sep=0cm,
  node distance=5mm}}

  \node (hm-label)  {\code{HashMap}};

  % First line linked list
  \node[n, sentinel, node distance=7mm] (n11) [below of=hm-label] {};
  \node[n]           (n12) [left of=n11] {};
  \node[n]           (n13) [left of=n12] {};
  % Second line linked list
  \node[n, sentinel] (n21) [below of=n11] {};
  \node[n]           (n22) [left of=n21] {};
  % Third line linked list
  \node[n, sentinel] (n31) [below of=n21] {};
  \node[n]           (n32) [left of=n31] {};
  \node[n]           (n33) [left of=n32] {};
  \node[n]           (n34) [left of=n33] {};
  % Fourth line linked list
  \node[n, sentinel] (n41) [below of=n31] {};

  % First line arrows
  \draw[-latex] (n11.west) -- (n12.east);
  \draw[-latex] (n12.west) -- (n13.east);
  \draw[-latex] (n13.south) |- ($(n13.south)!0.5!(n21.north)$) -| (n21.north);
  % Second line arrows
  \draw[-latex] (n21.west) -- (n22.east);
  \draw[-latex] (n22.south) |- ($(n22.south)!0.5!(n31.north)$) -| (n31.north);
  % Third line arrows
  \draw[-latex] (n31.west) -- (n32.east);
  \draw[-latex] (n32.west) -- (n33.east);
  \draw[-latex] (n33.west) -- (n34.east);
  \draw[-latex] (n34.south) |- ($(n34.south)!0.5!(n41.north)$) -| (n41.north);

  \node[arr, node distance=1.6cm]        (a1) [right of=n11] {};
  \node[arr]                             (a2) [below of=a1]  {};
  \node[arr]                             (a3) [below of=a2]  {};
  \node[arr]                             (a4) [below of=a3]  {};

  \draw[circlebase,-latex] (a1.center) -- (n11.east);
  \draw[circlebase,-latex] (a2.center) -| ($(a2.west)!0.66!(n31.east)$) |- (n31.east);
  \draw[circlebase,-latex] (a3.center) -- (n21.east);
  \draw[circlebase,-latex] (a4.center) -| ($(a4.west)!0.33!(n41.east)$) |- (n41.east);

  \node[inner sep=4mm, draw, fit={(n34) (n41) (hm-label) (a4)},
        label=above:Global Shared State] (border) {};

  \node[draw, node distance=4cm] (f1) [right of=hm-label] {\code{fn()}};
  \node[draw, node distance=2cm] (r1) [right of=f1] {\code{RandomSource}};
  \node[node distance=1.6cm]       (l1) [right of=r1] {$T_1$};
  \node[draw, fit={(f1) (l1)},label=above:Thread local mutable state] {};
  \draw[-latex] (f1.east) -- (r1.west);
  \draw[-latex] (f1.west) -- (f1 -| border.east);

  \node[draw, node distance=1.3cm] (f2) [below of=f1] {\code{fn()}};
  \node[draw, node distance=2cm] (r2) [right of=f2] {\code{RandomSource}};
  \node[node distance=1.6cm]       (l2) [right of=r2] {$T_2$};
  \draw[-latex] (f2.east) -- (r2.west);
  \draw[-latex] (f2.west) -- (f2 -| border.east);
  \node[draw, fit={(f2) (l2)}] {};

  \node[draw, node distance=1.3cm] (f3) [below of=f2] {\code{fn()}};
  \node[draw, node distance=2cm] (r3) [right of=f3] {\code{RandomSource}};
  \node[node distance=1.6cm]       (l3) [right of=r3] {$T_3$};
  \draw[-latex] (f3.east) -- (r3.west);
  \draw[-latex] (f3.west) -- (f3 -| border.east);
  \node[draw, fit={(f3) (l3)}] {};

  \node[draw, node distance=1.3cm] (f4) [below of=f3] {\code{fn()}};
  \node[draw, node distance=2cm] (r4) [right of=f4] {\code{RandomSource}};
  \node[node distance=1.6cm]       (l4) [right of=r4] {$T_4$};
  \draw[-latex] (f4.east) -- (r4.west);
  \draw[-latex] (f4.west) -- (f4 -| border.east);
  \node[draw, fit={(f4) (l4)}] {};

\end{tikzpicture}

  \caption{Illustration of the \code{HashMap} benchmark. The four threads all have their own
  \code{RandomSource} which supplies thread local random numbers, and they share the
  \code{HashMap}.}
\end{figure}

For CMR this is useful since we can put the data strucutres we want to
benchmark in the immutable shared state, as neither of the operations we want to test are
\code{\&mut self} (see \cref{sec:concurrency-and-aliasing}). The user specifies the function to be
benchmarked, the number of threads, and the states, and the duration of the benchmark, and
\code{trench} handles the rest. The number of runs of the function specified during the given
duration is measured.


\cref{lst:trench-ex} shows the benchmark for \code{HashMap::insert}.  \code{RandomSource} allows us
to pregenerate random numbers that we can insert into the hashmap, such that the random number
generation itself is thread local, and is not included in the benchmarking loop.

The \code{with_local_state} function runs the closure on each thread in parallel; this is used both
for initializing the local state, and for thread local initialization and destruction.
The global and local states, \code{HmState} and \code{RandomSource}, implements the trait
\code{Default}, so that we do not have to initialize it ourselves.
